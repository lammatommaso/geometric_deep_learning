\documentclass[../main.tex]{subfiles}
\begin{document}

\begin{prop}
    \underline{Variational problems related to the laplacian}\\
    Let $f \in L^2(M)$ the variational problem minimizing the Dirichlet energy functional $\int_M dx (gradf)^2$, 
    admit as solution the kernel of the laplacian. Otherwise if we wish to have normalized functions only
    to avoid uniformly vanishing functions i.e. $\int_M dx f^2 = 1$ we get as solution the eigenfunction of the laplacian
    relative to its lowest eigenvalue. After this the same process can be applied to graphs.
\end{prop}
\begin{proof} The proof is split in two different problems:
    \begin{itemize}
        \item 
	$\int f \Delta f = \int (gradf)^2$, the problem $\delta \int (gradf)^2 = 0$ is solved with the
	Euler-Lagrange equations $div \pder{\mc{L}}{(gradf)}=\pder{\mc{L}}{f}$, which give $\Delta f = 0$.
	\item
	This time we have $\delta[\int (gradf)^2 - \lambda (\int f^2 -1)]=0$, which using the same equations
	leads to $\Delta f = \lambda f$, and since for those functions $\int f \Delta f = \lambda$, we want to find
	the minumum eigenvalue of the laplacian.
    \end{itemize}
\end{proof}

\begin{prop}
    \underline{Discrete "variational" problems related to the laplacian}\\
    Let $\phi$ be an $k \times k$ matrix, the variational problem minimizing the Dirichlet energy $Tr(\phi^T \Delta \phi)$ is equivalent to
    the problem $\Delta \phi = 0$, while if we normalize the functions we get $\Delta \phi = \Lambda \phi$, where $\Lambda = diag(\lambda)$ with 
    the k lowest eigenvalues.
\end{prop}
\begin{proof} As follows:\\
    \begin{itemize}
        \item
	Let $min_{\phi}Tr(\phi^T \Delta \phi) = \phi_{li}\Delta_{lk}\phi_{ki} = \sum_i \phi_i^T \Delta \phi_i $ (Einstein notation) be our optimization problem,
	since $\Delta > 0$ we can simply solve for all $\phi_i$ and minimize $\pder{\mc{L}_i}{\phi_j} = \pder{\phi_i^T \Delta \phi_i}{\phi_j}
	 = 2\Delta\phi_i=0$.
	 \item
	Let $min_{\phi}Tr(\phi^T \Delta \phi) = \phi_{li}\Delta_{lk}\phi_{ki} = \sum_i \phi_i^T \Delta \phi_i $ (Einstein notation) be our optimization problem,
        under the constraint $\phi^T \phi = I$(orthonormal) since $\Delta > 0$ we can simply solve for all $\phi_i$ and minimize 
	$\pder{\mc{L}_i}{\phi_j} = \pder{(\phi_i^T \Delta \phi_i -\lambda_i(\phi_i^T\phi_i -1))}{\phi_j}$(non sostituisco l'1 perché non posso sostituire il vincolo 
	nel vincolo)$ = 2\Delta\phi_i-2\lambda_i\phi_i=0$. And since the initial trace is equal to the sum of our k positive eigenvalues we shall take the
	k lowest eigenvalues. The problem can be shown to be $\Delta \phi = \Lambda \phi$.
   \end{itemize}
\end{proof}

\begin{prop}
    \underline{Continuous spectrum of the laplacian in $\R^3$}\\
    Let $f \in L^2(\Omega)$, in particular in a Schwartz space where the Fourier Transform is invertible and selfadjoint,
    then we have that $\Delta \phi_p = p^2 \phi_p$, where $\forall f \in L^2$ we can write $f = \int_\Omega dp c_p \phi_p$
    and $\int_\Omega dx \phi_p \phi_{p'} = \delta(p-p')$, that is generalized Hilbert Base up to normalization.
\end{prop}
\begin{proof}:\\
    So if the Fourier transform is invertible we have that $f(x') = \int_\Omega dp e^{i\scal{x'}{p}}\int_\Omega dx e^{-i\scal{x}{p}}f(x)$, 
    then by the definition of the shift functional we can represent it with its integral kernel $\int_\Omega dp e^{i\scal{x'-x}{p}}$, which
    is equal to $\delta(x-x') = \delta(x'-x)$. Trivially we have that our laplacian eigenfunctions are $\phi_p = e^{i\scal{x}{p}}$ of eigenvalue $p^2$, since 
    $f(x) = \mc{F}^\dag \mc{F} = \int_\Omega dp (\mc{F}f)(p)e^{i\scal{x}{p}}$ we define our coefficients $c_p$. We just have to show that
    $\int dx e^{i\scal{x}{p'-p}} = \delta(p'-p)$ that is true by definition.
\end{proof}
    

% Secondo me la dimsotrazione con la Traccia è molto brutta
% Aggiungere le condizioni al contorno per il problema variazionale continuo(Fomin)


\end{document}
