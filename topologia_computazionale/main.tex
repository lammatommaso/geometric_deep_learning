\documentclass{article}
\usepackage[utf8]{inputenc}
\usepackage{fouriernc}
\usepackage[T1]{fontenc}
\usepackage{adjustbox}
\usepackage[margin=2cm]{geometry}
\usepackage{graphicx}
\usepackage{hyperref}
\usepackage{amsmath}
\usepackage{amsthm}
\usepackage{amssymb}
\hypersetup{colorlinks=true,linkcolor=blue,urlcolor=red}

\newtheorem{defn}{Definizione}[section]
\newtheorem{thm}{Teorema}[section]
\newtheorem*{thm*}{Teorema}
\newtheorem{prop}[thm]{Proposizione}
\newtheorem*{prop*}{Proposizione}
\newtheorem{lem}[thm]{Lemma}
\newtheorem{cor}{Corollario}[thm]
\newtheorem{exa}{Esempio}
\newtheorem{exe}{Esercizio}

\newcommand{\bb}[1]{\textbf{#1}}
\newcommand{\N}{\mathbb{N}}
\newcommand{\R}{\mathbb{R}}
\newcommand{\pder}[2]{\frac{\partial #1}{\partial #2}}
\newcommand{\eval}[1]{\Big{|}_{#1}}

\title{Appunti di Topologia Computazionale}
\author{Tommaso Lamma}
\begin{document}
\maketitle
L'importante non sono i dati ma la reazione dell'osservatore in relazione ai dati.\\
  \begin{section}{Lezione del 23/09/2020}
    \begin{defn}{\scshape Combinazione Affine\\}
      Una combinazione lineare $L=\sum_i \lambda_i x_i$ con $x_i \in \R^d$ e $\lambda_i \in \R$
      è detta \bb{combinazione affine} se $\sum_i \lambda_i=1$.
    \end{defn}
    \begin{defn}{\scshape Inviluppo Affine\\}
    \end{defn}
    \begin{defn}{\scshape Combinazione Convessa\\}
    \end{defn}
    \begin{defn}{\scshape Inviluppo Convesso\\}
    \end{defn}
    \begin{defn}{\scshape Affine Indipendenza\\}
    \end{defn}
    \begin{prop}%Aff.ind <-> Lin.ind
    \end{prop}
    \begin{cor}%Sull'affine dipendenza
    \end{cor}
    \begin{defn}{\scshape Combinazione Affine\\}
    \end{defn}
    \begin{defn}{\scshape Combinazione Affine\\}
    \end{defn}
    \begin{defn}{\scshape Combinazione Affine\\}
    \end{defn}
    \begin{defn}{\scshape Combinazione Affine\\}
    \end{defn}
  \end{section}
  \begin{section}{Lezione del 24/09/2020}
    \begin{defn}{\scshape k-Simplesso\\}
    \end{defn}
    \begin{defn}{\scshape Faccia di un k-Simplesso\\}
    \end{defn}
    \begin{defn}{\scshape Frontiera di un k-Simplesso\\}
    \end{defn}
    \begin{defn}{\scshape Interno di un k-Simplesso\\}
    \end{defn}
    \begin{defn}{\scshape Complesso Simpliciale Geometrico(finito)\\}
    \end{defn}
    \begin{defn}{\scshape Dimensione di un C.S.G.\\}
    \end{defn}
    \begin{defn}{\scshape Corpo di un C.S.G.\\}
    \end{defn}
    \begin{defn}{\scshape Sottocomplesso Simpliciale Geometrico\\}
    \end{defn}
    \begin{defn}{\scshape j-Scheletro di un C.S.G.\\}
    \end{defn}
    \begin{defn}{\scshape Stella di un C.S.G.\\}
    \end{defn}
    \begin{defn}{\scshape Stella Chiusa di un C.S.G.\\}
    \end{defn}
    \begin{defn}{\scshape Cintura di un C.S.G.\\}
    \end{defn}
    \begin{defn}{\scshape Complesso Simpliciale Astratto(finito)\\}
    \end{defn}
    \begin{defn}{\scshape Dimensione di un C.S.A.\\}
    \end{defn}
    \begin{defn}{\scshape Faccia Astratta\\}
    \end{defn}
    \begin{defn}{\scshape Insieme di vertici di un C.S.A.\\}
    \end{defn}
    \begin{defn}{\scshape Isomorfismo tra C.S.A.\\}
    \end{defn}
  \end{section}
  \begin{section}{Lezione del 29/09/2020}
    \begin{defn}{\scshape Schema di un C.S.G. e Realizzazione di un C.S.A.\\}
    \end{defn}
    \begin{thm}{\scshape Realizzazione Geometrica di un C.S.A.\\}
    \end{thm}
  \end{section}
  \begin{section}{Lezione del 30/09/2020}
    \begin{defn}{\scshape Grafo \\}
    \end{defn}
    \begin{thm}{\scshape Teorema di Kuratowski\\}
    \end{thm}
    \begin{thm}{\scshape Esistenza e Uncità delle Coordinate Baricentriche\\}
    \end{thm}
    \begin{thm}{\scshape Omeomorfismo tra i Corpi di Realizzazioni Geometriche dello stesso C.S.A.\\}
    \end{thm}
    \begin{defn}{\scshape Mappa di Vertici\\}
    \end{defn}
    \begin{defn}{\scshape Applicazione Simpliciale\\}
    \end{defn}
    \begin{defn}{\scshape Mappa PL(piecewise linear)\\}
    \end{defn}
    \begin{defn}{\scshape Categoria\\}
    \end{defn}
  \end{section}
\end{document}
