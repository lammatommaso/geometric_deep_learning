\documentclass{article}
\usepackage[utf8]{inputenc}
\usepackage{fouriernc}
\usepackage[T1]{fontenc}
\usepackage{adjustbox}
\usepackage[margin=2cm]{geometry}
\usepackage{graphicx}
\usepackage{hyperref}
\usepackage{amsmath}
\usepackage{amsthm}
\usepackage{amssymb}
\hypersetup{colorlinks=true,linkcolor=blue,urlcolor=red}

\newtheorem{thm}{Theorem}[section]
\newtheorem*{thm*}{Theorem}
\newtheorem{prop}[thm]{Proposition}
\newtheorem*{prop*}{Proposition}
\newtheorem{lem}[thm]{Lemma}
\newtheorem{cor}{Corollary}[thm]
\newtheorem{exa}{Example}
\newtheorem{exe}{Exercise}

\newcommand{\bb}[1]{\textbf{#1}}
\newcommand{\N}{\mathbb{N}}
\newcommand{\R}{\mathbb{R}}
\newcommand{\pder}[2]{\frac{\partial #1}{\partial #2}}
\newcommand{\eval}[1]{\Big{|}_{#1}}

\title{The additivity of ignorance}
\author{Tommaso Lamma}
\begin{document}
\maketitle
\begin{prop*}{\scshape Additivity of Ignorance\\}
\bb{Nulla di tutto ciò è legale ma ora so usare latex}\\

Let $\sigma :\N\to\R$  be a possible ignorance function, if $\sigma(1)=0$ and $\sigma(xy)=f(\sigma(x),\sigma(y))$,\\ 
where $f:\R\to\R$  is a differentiable function, then $\sigma(xy)=\sigma(x)+\sigma(y)$.
\end{prop*}
\begin{proof}
\begin{gather}
    \sigma(xy)=f(\sigma(x),\sigma(y))\\
    \pder{f}{x_1}=g(x_1)\\
    \pder{f}{x_2}=g(x_2)\\
    f=G(x_1)+c(x_2)+d_1\\
    f=G(x_2)+c(x_1)+d_2\\
    f=G(x_1)+G(x_2)+d\\
    \sigma(xy)=G(\sigma(x))+G(\sigma(y))+d\\
    \pder{\sigma(xy)}{x}\eval{y=1}=y\eval{y=1}\frac{d\sigma(x)}{dx}=\frac{dG(x_1)}{dx_1}\frac{dx_1}{dx}
        =\frac{dG(\sigma(x))}{d\sigma}\frac{d\sigma}{dx}\\
    \frac{dG(\sigma)}{d\sigma}=1\\
    G(\sigma)=\sigma+c\\
    \sigma(xy)=\sigma(x)+\sigma(y)+2c+d\\
    \sigma(xy)=\sigma(x)+\sigma(y)+C\\
    \sigma(x)=\sigma(1)+\sigma(x)+C=\sigma(x)+C\\
    \sigma(1)=0\rightarrow C=0\\
\end{gather}  
\end{proof}
\end{document}
