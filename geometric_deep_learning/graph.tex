\documentclass{article}
\usepackage[utf8]{inputenc}
\usepackage{fouriernc}
\usepackage[T1]{fontenc}
\usepackage{adjustbox}
\usepackage[margin=2cm]{geometry}
\usepackage{graphicx}
\usepackage{hyperref}
\usepackage{amsmath}
\usepackage{amsthm}
\usepackage{amssymb}
\usepackage{mathtools}
\hypersetup{colorlinks=true,linkcolor=blue,urlcolor=red}

\setlength{\parindent}{0pt}

\newtheorem{thm}{Theorem}[section]
\newtheorem*{thm*}{Theorem}
\newtheorem{prop}[thm]{Proposition}
\newtheorem*{prop*}{Proposition}
\newtheorem{defn}{Definition}[section]
\newtheorem*{defn*}{Definition}
\newtheorem{lem}[thm]{Lemma}
\newtheorem{cor}{Corollary}[thm]
\newtheorem{exa}{Example}
\newtheorem{exe}{Exercise}

\newcommand{\bb}[1]{\textbf{#1}}
\newcommand{\eps}{\varepsilon}
\newcommand{\N}{\mathbb{N}}
\newcommand{\R}{\mathbb{R}}
\newcommand{\pder}[2]{\frac{\partial #1}{\partial #2}}
\newcommand{\eval}[1]{\Big{|}_{#1}}

\title{Are Graphs Manifolds?}
\author{}
\date{}
\begin{document}
    \maketitle
    \begin{section}{Graphs}
    \begin{defn}
        Let $\mathcal{G}$ be a connected simplicial complex such that $dim\mathcal{G}=1$ and let $\mathcal{V}:=Vert\mathcal{G}$
	and $\mathcal{E}:=\mathcal{G}^{(i)}$, let $F,G:\mathcal{V} \to \mathcal{E}$ and $f,g:\mathcal{V} \to \R$ we want to define
	a scalar product on $L^2(\mathcal{E})$ and on $L^2(\mathcal{V})$:\\
	$<f,g>_{\mathcal{V}}=\sum_{i \in \mathcal{V}}a_if_ig_i$\\
	$<F,G>_{\mathcal{E}}=\sum_{\mathcal{E}}w_{ij}F_{ij}G_{ij}$ DO NOT COUNT THE EDGES TWICE!*
    \end{defn}
    \begin{defn}
        $grad: L^2(\mathcal{V} \to L^2(\mathcal{E})$ such that $f \mapsto f_i-f_j$\\
	$div: L^2(\mathcal{E}) \to L^2(\mathcal{V})$ such that $F \mapsto \frac{1}{a_i}\sum_{j:(i,j)\in\mathcal{E}}w_{ij}F_{ij}$
    \end{defn}
    \begin{prop}
        $<divF,f>_{\mathcal{V}}=<F,gradf>_{\mathcal{E}}$ if $F_{ij}=-F_{ji}$
    \end{prop}
    \begin{proof}
        $\sum_{i \in \mathcal{V}}a_if_idivF_i=\sum_{\mathcal{E}}w_{ij}F_{ij}(f_i-f_j)
	=*\sum_{i,j:(i,j)\in\mathcal{E}}w_{ij}F_{ij}f_i$\\
	then $a_idivF_i=\sum_{j:(i,j)\in\mathcal{E}}w_{ij}F_{ij}$
    \end{proof}
    \begin{defn}
        $\partial\mathcal{A \subset V}:=\{(i,j)\in\mathcal{E}:i\in{\mathcal{A}}, j\notin{\mathcal{A}}\}$ 
	the second condition is equivalent to outwards oriented surface
    \end{defn}
    \begin{prop}
        $\sum_{i\in\mathcal{A}}divF_i=\sum_{i,j:(i,j)\in\partial\mathcal{A}}F_{ij}$ assuming $a_i=w_{ij}=1$
    \end{prop}
    \begin{proof}
        $\sum_{i,j:i\in\mathcal{A},(i,j)\in\mathcal{E}}=\sum_{i,j:i\in\mathcal{A},(i,j)\in\mathcal{E},j\notin\mathcal{A}}
	+\sum_{i,j:i\in\mathcal{A},(i,j)\in\mathcal{E},j\in\mathcal{A}}$\\
	In the second sum the simmetric adjacency matrix kills the antisimmetric $F_{ij}$, so only the first one remains
    \end{proof}
    \end{section}
\end{document}
