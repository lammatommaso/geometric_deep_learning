\documentclass[../2.tex]{subfiles}
\begin{document}

We now proceed by analogy with the previous section and define a \ii{discrete Dirichlet energy} on graphs, using the laplacian matrix written on the canonical basis, as
in definition \ref{lapdef}.
Our reference for this part is \cite{bronstein}.

\begin{defn}
    Let $A,\Phi$ be to $n\times n$ matrices, we define the \ii{discrete Dirichlet energy} to be 
    \[ D[\Phi] = \tr(\Phi^T \Delta_0 \Phi). \]
\end{defn}

Notice the analogy with the setting as expressed by proposition \ref{prop:2:4:3}.
In the following treatment we will simply use $\Delta$ for the $0$-laplacian.

We have already seen that the $0$-laplacian can be represented as a matrix with respect to the canonical basis
of $0$-chains. Let us now see a more efficient way to compute the $0$-laplacian matrix. In order to do that we need to introduce some notation.

\begin{defn}
    Let $\mc{G}$ be a simple undirected graph, let $\ch{i} \in C_0$ we define the \ii{degree} of $\ch{i}$ as 
    \[ \dg(i) := \sum_{j : \ch{i,j} \in C_1} 1. \]
\end{defn}

According to this definition the degree of a vertex of a graph is the number of vertices that vertex is linked to by an edge(for the notation $\ch{i}$ see app. \ref{app:B}).

\begin{prop}
    The laplacian matrix on the canonical basis of $0$-chains can be written as
    \[ \Delta = D - A, \]
    where $D_{ij} := \delta_{ij}deg(i)$ is the degree matrix and $A$ is the adjacency matrix.
    \label{prop:d-a}
\end{prop}
\begin{proof}
    This proposition is an immediate consequence of definition \ref{def:2:2:2}. \qedhere
\end{proof}

The construction of this same matrix in the example \ref{ex:2:3:2} was presented to become acquainted with the abstract definition of laplacian operators.
Proposition \ref{prop:d-a} actually gives the most efficient way to write the $0$-laplacian matrix for any simple undirected graph.

It is easy to see that graphs with multiple connected components the $0$-laplacian matrix is a block matrix whose blocks are the laplacians of the connected components.

Let us now focus on a connected graph or in alternative on a single connected component of an unconnected graph. 
The $i$-th column of the matrix $\Delta$ has $\dg{(i)}$ on the $i$-th row and $-1$ on the $j$-th row if $\ch{i,j} \in C_1$. If we recall the definition of degree of a vertex
it is immediate to understand that the sum of the elements of a column is always zero, therefore the sum of the rows is also zero. From these considerations
we derive the following proposition.

\begin{prop}
    Let $\mc{G}$ be a simple undirected connected graph then 
    \[ \sum_{C_0} \ch{i} \in \ker\Delta. \]
\end{prop}

Since according to Eckmann's theorem the dimension of the kernel of the $0$-laplacian for a simple connected graph is one, 
we have that $\sum_{C_0} \ch{i}$ is the only linearly independent chain in the kernel, therefore the unique $0$-laplacian eigenfunction
with eigenvalue zero up to scalar multiplication. Since the coefficients in the formal sum of the basis $0$-chains are all equal, we have that the $0$-laplacian eigenchain
with eigenvalue zero of a simple undirected connected graph is always uniform.\\
\hfill \\

The minimization of the discrete Dirichlet energy is not a variational problem, rather a simple minimization in $\Phi$ in which we can 
use the standard Lagrange multipliers theorem. Here the $0$-laplacian is a matrix instead of a differential operator. 
\begin{prop}
    The minimum for the Dirichlet energy $ \tr(\Phi^T \Delta \Phi)$ constrained to the $\Phi$ such that $\Phi^T \Phi = 1$ is reached when the
    rows of $\Phi$ are the laplacian eigenvectors.
\end{prop}
\begin{proof}
    The Lagrange multipliers' theorem allows us to use constraints by minimizing the following constrained Dirichlet energy
    \[ D[\Phi] = \sum_{i,j,k} \Phi_{ik} \Delta_{ij} \Phi_{jk} - \sum_{i,j,k} \Lambda_{ij} \Phi_{kj} \Phi_{ki}, \]
    where $\Lambda$ is a diagonal matrix with the Lagrange multipliers as eigenvalues.\\
    In fact the constraints $\Phi^T \Phi = 1$ can be written as $\sum_k \Phi_{kj} \Phi{ki} = \delta_{ij}$, therefore we need to subtract from the lagrangian
    the quantity $\sum_{i,j} \sum_k \Lambda_{ij} \Phi_{kj} \Phi{ki}$. Actually because of the constraint itself we already know that for $i \neq j$ the quantity
    we subtracted from the lagrangian vanishes, therefore we can see $\Lambda$ as a diagonal matrix of lagrange multipliers.  

    We then minimize this energy
    \[ \frac{\del D[\Phi]}{\del \Phi_{mn}} = \sum_{i,j,k} \delta_{im}\delta_{nk}\Delta_{ij} \Phi_{jk} + \sum_{i,j,k} \Phi_{ik} \Delta_{ij}\delta_{jm}\delta_{nk}
    - \sum_{i,j,k} \Lambda_{ij} \delta_{mk}\delta_{nj}\Phi_{ki} - \sum_{i,j,k} \Lambda_{ij} \Phi_{kj} \delta_{mk}\delta_{ni} =  \]
    \[ = \sum_{j} \Delta_{mj} \Phi_{jn} + \sum_{i} \Phi_{in} \Delta_{im} - \sum_{i} \Lambda_{in} \Phi_{mi} - \sum_{j} \Lambda_{nj} \Phi_{mj} = 0, \]
    which using the symmetry of $\Lambda$ and $A$ can be written as
    \[ \sum_{j} \Delta_{mj} \Phi_{jn} + \sum_{i}  \Delta_{mi} \Phi_{in} - \sum_{i} \Phi_{mi}\Lambda_{in} - \sum_{j} \Phi_{mj} \Lambda_{jn} =  \]
    \[ 2\Delta\Phi - 2\Phi\Lambda = 0. \]
    We then obtain 
    \[ \Delta\Phi = \Phi\Lambda, \]
    which for every row of $\Phi$ is an eigenvalue equation. \qedhere
\end{proof}

\begin{figure}[H]
    \centering
    \includegraphics[width=17cm, height=4cm]{sections/2/eiglap}
    \caption{A graphical representation of the first laplacian eigenchains on the Minnesota Road Graph.
    The color of a node $\ch{i}$ represents the value $\braket{i}{\phi_j}$ of the $0$-laplacian eigenfunction $\ch{\phi_j}$
    on the node $\ch{i}$.}
    \label{fig:2:5}
\end{figure}

In figure \ref{fig:2:5} we can see some of the $0$-laplacian eigenfunctions. We notice that the eigenfunction corresponding to the eigenvalue $0$ is uniform.
   
\end{document}