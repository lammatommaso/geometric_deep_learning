\documentclass[../2.tex]{subfiles}
\begin{document}
    In \autoref{ch:1} we studied homology and cohomology on abstract simplicial complexes, in this chapter we shall discuss in detail
    $1$-dimensional abstract simplicial complexes.

    \begin{defn}
        A \ii{graph} is a $1$-dimensional abstract simplicial complex.
    \end{defn}

    In a graph $\mc{G}$ the non trivial chain groups are therefore $C_0(\mc{G})$ and $C_1(\mc{G})$, also called vertex and edge spaces. 
    The canonical basis of the vertex space is the set of vertexes $\{\ket{e_i}\}_{i \in I}$, and that of edges $\{\ket{e_i,e_j} : e_i \sim e_j\}$, where $e_i \sim e_j$ means $e_i$ is adjacent to $e_j$.

    \begin{defn}
        Let $\mc{G}$ be a graph, we define the \ii{adjacency matrix} $A$ of $\mc{G}$ on the canical vertex basis by
        \[a_{ij} = 
        \begin{cases}
            1 & e_i \sim e_j \\
            0 & otherwise \\
        \end{cases}
        \]
        
        
    \end{defn}
    Only one boundary and one coboundary can be defined on graphs, we shall name those gradient and divergence respectively.

    \begin{defn}
        Let $\mc{G}$ be a graph, we define the \ii{gradient} $\del_1 : C_1 \to C_0$ by 
        \[ \del_1\ket{e_i,e_j} := e_i - e_j \quad \forall \ket{e_i,e_j} \in C_1.\]
    \end{defn}
    
    \begin{defn}
        Let $\mc{G}$ be a graph, we define the \ii{divergence} $\del^\dagger_1 : C_0 \to C_1$ by 
        \[ d_1\ket{e_i} := \sum_{\ket{e_i,e_j} : x_i \sim x_j} \ket{e_i,e_j}.\]
    \end{defn}

    {\color{red} Disegno che mostra il cobordo di un nodo}

    The prevously defined generalizd Stokes' theorem is on graphs a discrete version of the divergence theorem.

    \begin{thm}
        Let $\mc{G}$ be a graph, let $x \in C_0, v \in C_1$ then $\bra{x}\del^\dagger\ket{v} = \bra{v}\del\ket{x}$.
    \end{thm}

    {\color{red} Disegno che mostra il teorema della divergenza discreto}

    On a graph that admits non trivial gradient and divergence, we can define a non trivial laplacian operator as the gradient of the divergence.
    Although this might sound strange, the inversion of gradient and divergence in the definition of laplacian is due to the fact that we are representing
    the functions on the graph (cochains) by their dual chains.

    \begin{defn}
        Let $\mc{G}$ be a graph, we define the \ii{laplacian} $\Delta_1 : C_0 \to C_0$ by 
        \[ \Delta_1 := \del_1 \circ \del^\dagger_1.\]
    \end{defn}

    An intersting property of the laplacian is that the dimension of its kernel equals the connected components of the graph.

    {\color{red} Esempio con un grafo piccolo e disegno \\

    Let $G$ be the graph illustrated 
    
    then the laplacian expressed in terms of the canoncal vertex basis is 

    \[\begin{pmatrix}
            2 & -1 & -1 & 0 & 0 & 0 & 0 & 0 & 0 \\
            -1 & 2 & -1 & 0 & 0 & 0 & 0 & 0 & 0 \\
            -1 & -1 & 2 & 0 & 0 & 0 & 0 & 0 & 0 \\
            0 & 0 & 0 & 1 & -1 & 0 & 0 & 0 & 0 \\
            0 & 0 & 0 & -1 & 3 & -1 & -1 & 0 & 0 \\
            0 & 0 & 0 & 0 & -1 & 1 & 0 & 0 & 0 \\
            0 & 0 & 0 & 0 & -1 & 0 & -1 & 0 & 0 \\
            0 & 0 & 0 & 0 & 0 & 0 & 0 & 1 & -1 \\
            0 & 0 & 0 & 0 & 0 & 0 & 0 & -1 & 1 
        \end{pmatrix} \] 

    Three invariant subspaces emerge from the laplacian, that determine three different laplacians, namely
    \[ \Delta = \Delta_A \oplus \Delta_B \oplus \Delta_C \]

    Furthermore any of those three blocks has a 1-dimensional kernel.
        
        
        }

    This theorem will be proven in \autoref{ch:3}.
\end{document}