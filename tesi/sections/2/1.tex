\documentclass[../1.tex]{subfiles}
\begin{document}

    An important role in the definition of a convolution on simplicial complexes is played by the Laplacian operator,
    especially by its eigenfunctions and spectrum. A more thorough discussion of this operator can be found in \cite{laplacian}.
    
    \begin{defn}
        We define the \ii{p-Laplacian} operator to be 
        \[ \Delta_p := \del_{p+1}\del^\dagger_{p+1} + \del^\dagger_p\del_p =: \Delta^+_p + \Delta^-_p.\]
    \end{defn}

    The Laplacian operator is defined to be self-adjoint and positive definite.

    \begin{prop}
        Let $\Delta_p$ be a laplacian operator, then $\Delta^\dagger_p = \Delta_p.$
    \end{prop}
    \begin{proof}
        Let $\ket{\sigma}, \ket{\tau} \in C_p$ 
        \[ \bra{\sigma}\Delta_p\ket{\tau}=\bra{\sigma}(\del_{p+1}\del^\dagger_{p+1} + \del^\dagger_p\del_p)\ket{\tau} = \]
        \[ = \bra{\tau}(\del_{p+1}\del^\dagger_{p+1} + \del^\dagger_p\del_p)^\dagger\ket{\sigma} = \]
        \[ = \bra{\tau}(\del_{p+1}\del^\dagger_{p+1} + \del^\dagger_p\del_p)\ket{\sigma} = \bra{\sigma}\Delta_p\ket{\tau}. \qedhere \]
    \end{proof}

    According to the spectral theorem there exists a basis of eigenchains of the Laplacian, and since all $\Delta_p, \Delta^+_p, \Delta^-_p$
    are self-adjoint we can say that they all admit a basis of eigenchains.

    \begin{prop}
        Let $\Delta_p\ket{\sigma} = \lambda_\sigma\ket{\sigma}$ then $\lambda_\sigma \geq 0$.
    \end{prop}
    \begin{proof}
        Let $\Delta^+_p\ket{\sigma} = \lambda^+_\sigma \ket{\sigma}$, we see that $\bra{\sigma}\Delta^+_p\ket{\sigma} = | \del_{p+1}\ket{\sigma} |^2 \geq 0$, and since
        $\bra{\sigma}\Delta^+_p\ket{\sigma} = \lambda^+_\sigma\braket{\sigma}{\sigma}$ we have that $\lambda+_\sigma \geq 0.$\\
        Let then $\Delta^-_p\ket{\sigma} = \lambda^-_\sigma \ket{\sigma}$, we see that $\bra{\sigma}\Delta^-_p\ket{\sigma} = | \del_p \ket{\sigma} |^2 \geq 0$, and since
        $\bra{\sigma}\Delta^-_p\ket{\sigma} = \lambda^-_\sigma\braket{\sigma}{\sigma}$, we also have that $\lambda-_\sigma \geq 0.$\\
        Furthermore, since $\Delta^+_p\Delta^-_p = \Delta^-_p\Delta^+_p = 0$ we have that $[\Delta^+_p,\Delta^-_p] = 0$, thence
        $[\Delta_p, \Delta^\pm_p] = 0$, therefore $\Delta_p, \Delta^+_p, \Delta^-_p$ share a basis of eigenchains.
        Let $\ket{\sigma}$ be in that common basis then $\Delta_p\ket{\sigma} = \lambda_\sigma\ket{\sigma}$, where
        $\lambda_\sigma = \lambda^+_\sigma + \lambda^-_\sigma \geq 0$. \qedhere
    \end{proof}

    Another really interesting property that was first proven by Beno Eckmann in 1944, is that the kernel of the p-Laplacian
    is isomorphic to the p-homology group.

    \begin{thm}
        Let $\Delta_p$ be a laplacian operator, then $ker\Delta_p \simeq H_p$.
    \end{thm}
    \begin{proof}
        {\color{red}We have \[ \Delta_p =: \del_{p+1}\del^\dagger_{p+1} + \del^\dagger_p\del_p =: \Delta^+_p + \Delta^-_p.\]
        Because of the homology lemma $\Delta^+_p\Delta-_p = \Delta^-_p\Delta^+_p = 0$, therefore $ker\Delta^{\pm}_p \subset im\Delta^{\mp}_p.$\\
        It is trivial to see that, since $ker\Delta_p = ker\Delta^+_p \cap ker\Delta^-_p $, we have that $ker\del^\dagger_{p+1} \cap ker\del_p \subset ker\Delta_p.$ \\
        Less trivial is the opposite inclusion
        \[ \del^\dagger_{p+1}\sigma \in im\del^\dagger_{p+1},\sigma \in ker\Delta_p \implies \del^\dagger_{p+1}\sigma \in ker\del_{p+1} = (im\del^\dagger_{p+1})^\perp,\]
        \[ \del_p\sigma \in im\del_p,\sigma \in ker\Delta_p \implies \del_p\sigma \in ker\del^\dagger_p= (im\del_p)^\perp,\]
        \[ \del_p\sigma = \del^\dagger_{p+1}\sigma \in im\del^\dagger_{p+1} \cap (im\del^\dagger_{p+1})^\perp = im\del_p \cap (im\del_p)^\perp = 0,\]
        therefore $ker\Delta_p \subset ker\del^\dagger_{p+1} \cap ker\del_p $. Finally we notice that
        \[ ker\Delta_p = ker\del^\dagger_{p+1} \cap ker\del_p = (im\del_{p+1})^\perp \cap ker\del_p \simeq H_p. \qedhere\]}
    \end{proof}

\end{document}
