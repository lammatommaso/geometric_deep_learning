\documentclass[../2.tex]{subfiles}
\begin{document}
    Only one boundary $\del_1$ and one coboundary $\del_1^\dagger$ can be defined on graphs, we shall name their dual representatives  respectively gradient $d_1^\dagger$ and divergence $d_1$.

    \begin{defn}
        Let $\mc{G}$ be a graph, we define the \ii{boundary} $\del_1 : C_1 \to C_0$ by 
        \[ \del_1\ch{i,j} := \ch{i} - \ch{j} \quad \forall \ch{i,j} \in C_1.\]
    \end{defn}

    \begin{defn}
        Let $\mc{G}$ be a graph, we define the \ii{coboundary} $\del^\dagger_1 : C_0 \to C_1$ by 
        \[ \del^\dagger_1\ket{i} := \sum_{j : \ket{i,j} \in C_1} \ket{i,j}.\]
    \end{defn}

    From definition \ref{dualrel} we have that the divergence satisfies the dual relation
    \[ d_1\cc{i,j} = \cc{i,j}\del_1,\]
    where the cochain $\cc{i,j}\del_1$ is the dual of the chain $\del_1^\dagger\ch{i,j}$, 
    therefore we can represent the action of $d_1$ on cochain via the action of $\del_1^\dagger$ on chains.\\
    Similarly we can represent the action of $d_1^\dagger$ with $\del_1$.

    % {\color{red} Disegno che mostra il cobordo di un nodo}

    % The prevously defined generalizd Stokes' theorem is on graphs a discrete version of the divergence theorem.

    % \begin{thm}
    %     Let $\mc{G}$ be a graph, let $\ket{\phi} \in C_0, \ket{\psi} \in C_1$ then $\bra{\phi}\del_1\ket{\psi} = \bra{\psi}\del_1^\dagger\ket{\phi} = (d_1\bra{\phi})\ket{\psi}$.
    % \end{thm}

    % Therefore the integral of the cochain $\ket{\psi}$ over the boundary of $\ket{\phi}$ equals the integral of the divergence of $\ket{\psi}$ over $\ket{\phi}$.

    % {\color{red} Disegno che mostra il teorema della divergenza discreto}

\end{document}