\documentclass[../2.tex]{subfiles}
\begin{document}
In this section we will present some properties of the eigenfunctions of the previously defined laplacian operator.
The laplacian eigenfunction are the solution of a constrained variational problem for an action called \ii{Dirichlet energy}, as shown in \cite{bronstein}.

\begin{defn}
    Let $f:\R^n \to \R$ be a differentiable funcion with compact support, we define the \ii{Dirichlet energy functional} to be
    \[ D[f] := \int_{\R^n} dx |\nabla f|^2 .\]
\end{defn}

\begin{prop}
    If $f:\R^n \to\R$ has compact support or vanishes at infinity we have that
    \[ D[f] := \int_{\R^n} dx |\nabla f|^2 = \int_{\R^n} dx f\Delta f .\]
\end{prop}
\begin{proof}
    By recalling the definition of laplacian in $\R^n$ as $\Delta f := -\nabla \cdot \nabla f$, and we integrate by parts
    \[ 0 = \int_{\R^n} dx \nabla \cdot (f \nabla f) = \int_{\R^n} dx \nabla f \cdot \nabla f + \int_{\R^n} dx f \nabla \cdot \nabla f = \int_{\R^n} dx \nabla f \cdot \nabla f - \int_{\R^n} dx f \Delta f, \]
    hence the thesis. \qedhere
\end{proof}

The variational problem for the Dirichlet energy leads to the Euler-Lagrange equations when it is restricted to those functions which
vanish on the boundary of the domain of integration, which is true in the case of functions with compact support.
Further details and proofs of the following two propositions can be found in \cite{fomin} at $35$ and $12$ respectively.

\begin{prop}
    Let $f:\R^n \to \R$ be a sufficiently regular function with compact support, and $\mc{L}(f,\nabla f) : \R^n \to \R$ a sufficiently regular lagrangian,
    then the solutions of the variational problem 
    \[ \delta\int_{\R^n} dx \mc{L}(f,\nabla f) = 0 \]
    are the solutions of the differential equation 
    \[ \nabla \cdot \frac{\del \mc{L}}{\del(\nabla f)} - \frac{\del\mc{L}}{\del f} = 0. \]
\end{prop}

In order to solve the variational problem subject to constraints we need a variational 
version of the Lagrange multipliers theorem.

\begin{prop}
    Let $f:\R^n \to \R$ be a sufficiently regular function with compact support, and $\mc{L}(f,\nabla f) : \R^n \to \R$ a sufficiently regular lagrangian,
    then the solutions of the variational problem 
    \[ \delta\int_{\R^n} dx \mc{L}(f,\nabla f) = 0 \]
    constrained to the functions such that 
    \[ \int_{\R^n} dx f^2 = 1 \]
    are the solutions of the Euler-Lagrange equations for the lagrangian
    \[ \mc{L}(f,\nabla f) - \lambda f^2, \]
    where $\lambda$ is a Lagrange multiplier.
\end{prop}

We now apply this to the Dirichlet energy.

\begin{prop}
    The solution of the variational problem for the Dirichlet energy
    \[ \delta\int_{\R^n} dx | \nabla f |^2 = 0 \]
    constrained to the functions such that 
    \[ \int_{\R^n} dx f^2 = 1 \]
    are the eigenfunctions of the laplacian.
\end{prop}
\begin{proof}
    For this lagrangian the Euler-Lagrange equation reads as $ 2 \Delta f - 2\lambda f = 0$,
    hence 
    \[ \Delta f = \lambda f . \qedhere \]

\end{proof}

Obviously, the eigenfunction that truly minimizes the Dirichlet energy is the one corresponding to the lowest eigenvalue,
since for the eigenfunctions $D[f] = \int_{\R^n} dx f \Delta f = \lambda \int_{\R^n} dx f^2 = \lambda$.\\
\hfill \\

Another interesting property of the laplacian eigenfunctions comes from quantum mechanical considerations.
In fact, the laplacian, neglecting the constants, is the hamiltonian of the free particle in the Schr\"{o}dinger representation.

\begin{prop}
    The functions $e^{ip\cdot x} : \R^n \to \C$, where $p \in R^n$, are a generalized orthogonal basis of laplacian eigenfunctions.
\end{prop}
\begin{proof} 
    First we see that $\Delta e^{ip\cdot x} = \nabla \cdot (ipe^{ip\cdot x}) = -p^2 e^{ip\cdot x}$.
    Then, in a suitable space where the Fourier transform is invertible we have that $f(x') = \int_{\R^n} dp e^{i\scal{x'}{p}}\int_{\R^n} dx e^{-i\scal{x}{p}}f(x)$, 
    \\where $\int_{\R^n} dp e^{i\scal{x'-x}{p}} = \delta(x-x')$ up to constants that can be reabsorbed in the eigenfunctions, hence the generalized completness relation (see App. \ref{app:B}). \qedhere
\end{proof}

In this approach the Fourier transorm is nothing more than a change of generalized basis into the basis of the eigenfunctions of the laplacian.

Let's recall the definition of convolution in $\R^n$.

\begin{defn}
    Let $f,g : \R^n \to \C$ be sufficiently regular functions we define the \ii{convolution} of $f$ with $g$ as
    \[ (f * g)(x) = \int_{\R^n} dx' f(x')g(x-x').\]
\end{defn}

The well known theorem that follows states that the Fourier transform, as a change of generalized basis, diagonalizes the convolution.

\begin{thm}
    Let $\mc{F}$ denote the fourier transform, then for sufficiently regular functions $f,g : \R^n \to \C$ we have that
    \[ \mc{F}(f * g) = (\mc{F}f)(\mc{F}g).\]
\end{thm}

This approach will allow us in Chapter \ref{ch:3} to define a similar convolution as an operator which is diagonalized by the laplacian eigenfunctions.
    
\end{document}