\documentclass[../2.tex]{subfiles}
\begin{document}
In this section we will present some properties of the eigenfunctions of the previously defined laplacian operator.
The laplacian eigenfunction are the solution of a constrained variational problem for an action called \ii{Dirichlet energy}, as shown in \cite{bronstein}.

\begin{defn}
    Let $f:\R^n \to \R$ be a compact support differentiable funcion, we define the \ii{Dirichlet energy functional} to be
    \[ D[f] := \int_{\R^n} dx |\nabla f|^2 .\]
\end{defn}

\begin{prop}
    If $f:M\to\R$ vanishes on the boundary $\del M$ of the manifold, then we have that
    \[ D[f] := \int_{\R^n} dx |\nabla f|^2 = \int_{\R^n} dx f\Delta f .\]
\end{prop}
\begin{proof}
    By recalling the definition of laplacian in $\R^n$ as $\Delta f := -\nabla \cdot \nabla f$, and we integrate by parts
    \[ \int_{\R^n} dx \nabla \cdot (f \nabla f) = \int_{\R^n} dx \nabla f \cdot \nabla f + \int_{\R^n} dx f \nabla \cdot \nabla f = \int_{\R^n} dx \nabla f \cdot \nabla f - \int_{\R^n} dx f \Delta f = 0, \]
    hence the thesis. \qedhere
\end{proof}

The variational problem for the Dirichlet energy leads to the Euler-Lagrange equations when it is restricted to those functions which
vanish on the boundary of the domain of integration, which is true in the case of compact support functions.

\begin{prop}
    Let $f:\R^n \to \R$ be a sufficiently regular compact support function, and $\mc{L}(f,\nabla f) : \R^n \to \R$ a sufficiently regular lagrangian,
    then the solutions of the variational problem 
    \[ \delta\int_{\R^n} dx \mc{L}(f,\nabla f) = 0 \]
    are the solutions of the differential equation 
    \[ \nabla \cdot \frac{\del \mc{L}}{\del(\nabla f)} - \frac{\del\mc{L}}{\del f} = 0. \]
\end{prop}

In order to solve the variational problem subject to constraints we need a variational 
version of the Lagrange multipliers theorem.

\begin{prop}
    Let $f:\R^n \to \R$ be a sufficiently regular compact support function, and $\mc{L}(f,\nabla f) : \R^n \to \R$ a sufficiently regular lagrangian,
    then the solutions of the variational problem 
    \[ \delta\int_{\R^n} dx \mc{L}(f,\nabla f) = 0 \]
    constrained to the functions such that 
    \[ \int_{\R^n} dx f^2 = 1 \]
    are the solutions of the Euler-Lagrange equations for the lagrangian
    \[ \mc{L}(f,\nabla f) - \lambda f^2, \]
    where $\lambda$ is a Lagrange multiplier.
\end{prop}

Further details and proofs of the previous two propositions can be found in \cite{fomin} at $35$ and $12$ respectively.

\begin{prop}
    The solution of the variational problem for the Dirichlet energy
    \[ \delta\int_{\R^n} dx | \nabla f |^2 = 0 \]
    constrained to the functions such that 
    \[ \int_{\R^n} dx f^2 = 1 \]
    are the eigenfunctions of the laplacian.
\end{prop}
\begin{proof}
    For this lagrangian the Euler-Lagrange equation reads as $ 2 \Delta f - 2\lambda f = 0$,
    hence 
    \[ \Delta f = \lambda f . \qedhere \]

\end{proof}

Obviously, the eigenfunction that truly minimizes the Dirichlet energy is the one corresponding to the lowest eigenvalue,
since for the eigenfunctions $D[f] = \int_{\R^n} dx f \Delta f = \lambda \int_{\R^n} dx f^2 = \lambda$.
\hfill \\
We can define a \ii{discrete Dirichlet energy} on graphs, using the laplacian matrix written on the canonical basis.

\begin{defn}
    Let $A,\Phi$ be to $n\times n$ matrices, we define the \ii{discrete Dirichlet energy} to be 
    \[ D[\Phi] = \tr(\Phi^T \Delta \Phi). \]
\end{defn}

The minimization of the discrete Dirichlet energy is not a variational problem, rather a simple minimization in $\Phi$ in which we can 
use the standard Lagrange multipliers theorem.

\begin{prop}
    The minimum for the Dirichlet energy $ \tr(\Phi^T \Delta \Phi)$ constrained to the $\Phi$ such that $\Phi^T \Phi = 1$ is reached when the
    rows of $\Phi$ are the laplacian eigenfunctions.
\end{prop}
\begin{proof}
    The constrained action can also be written as
    \[ D[\Phi] = \sum_{i,j,k} \Phi_{ik} \Delta_{ij} \Phi_{jk} - \sum_{i,j,k} \Lambda_{ij} \Phi_{kj} \Phi_{ki}, \]
    where $\Lambda$ is a diagonal matrix with the Lagrange multipliers as eigenvalues.\\ We then minimize this action 
    \[ \frac{\del \mc{L}}{\del \Phi_{mn}} = \sum_{i,j,k} \delta_{im}\delta_{nk}\Delta_{ij} \Phi_{jk} + \sum_{i,j,k} \Phi_{ik} \Delta_{ij}\delta_{jm}\delta_{nk}
    - \sum_{i,j,k} \Lambda_{ij} \delta_{mk}\delta_{nj}\Phi_{ki} - \sum_{i,j,k} \Lambda_{ij} \Phi_{kj} \delta_{mk}\delta_{ni} =  \]
    \[ = \sum_{j} \Delta_{mj} \Phi_{jn} + \sum_{i} \Phi_{in} \Delta_{im} - \sum_{i} \Lambda_{in} \Phi_{mi} - \sum_{j} \Lambda_{nj} \Phi_{mj} = 0, \]
    which using the symmetry of $\Lambda$ and $A$ can be written as
    \[ \sum_{j} \Delta_{mj} \Phi_{jn} + \sum_{i}  \Delta_{mi} \Phi_{in} - \sum_{i} \Phi_{mi}\Lambda_{in} - \sum_{j} \Phi_{mj} \Lambda_{jn} =  \]
    \[ 2\Delta\Phi - 2\Phi\Lambda = 0. \]
    We then obtain 
    \[ \Delta\Phi = \Phi\Lambda, \]
    which for every row of $\Phi$ is an eigenvalue equation. \qedhere
\end{proof}

    
\end{document}