\documentclass[../1.tex]{subfiles}
\begin{document}

    An important role in the definition of a convolution on simplicial complexes is played by the Laplacian operator,
    especially by its eigenfunctions and spectrum. A more thorough discussion of this operator can be found in \cite{laplacian}.
    
    \begin{defn}
        We define the \ii{p-Laplacian} operator to be 
        \[ \Delta_p := \del_{p+1}\del^\dagger_{p+1} + \del^\dagger_p\del_p =: \Delta^+_p + \Delta^-_p.\]
    \end{defn}

    The Laplacian operator is defined to be self-adjoint and positive definite.

    \begin{prop}
        Let $\Delta_p$ be a laplacian operator, then $\Delta^\dagger_p = \Delta_p.$
    \end{prop}
    \begin{proof}
        Let $\ket{\sigma}, \ket{\tau} \in C_p$ 
        \[ \bra{\sigma}\Delta_p\ket{\tau}=\bra{\sigma}(\del_{p+1}\del^\dagger_{p+1} + \del^\dagger_p\del_p)\ket{\tau} = \]
        \[ = \bra{\tau}(\del_{p+1}\del^\dagger_{p+1} + \del^\dagger_p\del_p)^\dagger\ket{\sigma} = \]
        \[ = \bra{\tau}(\del_{p+1}\del^\dagger_{p+1} + \del^\dagger_p\del_p)\ket{\sigma} = \bra{\tau}\Delta_p\ket{\sigma}. \qedhere \]
    \end{proof}

    According to the spectral theorem there exists a basis of eigenchains of the Laplacian, and since all $\Delta_p, \Delta^+_p, \Delta^-_p$
    are self-adjoint we can say that they all admit a basis of eigenchains.

    \begin{prop}
        Let $\Delta_p\ket{\sigma} = \lambda_\sigma\ket{\sigma}$ then $\lambda_\sigma \geq 0$.
    \end{prop}
    \begin{proof}
        Let $\Delta^+_p\ket{\sigma} = \lambda^+_\sigma \ket{\sigma}$, we see that $\bra{\sigma}\Delta^+_p\ket{\sigma} = | \del_{p+1}\ket{\sigma} |^2 \geq 0$, and since
        $\bra{\sigma}\Delta^+_p\ket{\sigma} = \lambda^+_\sigma\braket{\sigma}{\sigma}$ we have that $\lambda+_\sigma \geq 0.$\\
        Let then $\Delta^-_p\ket{\sigma} = \lambda^-_\sigma \ket{\sigma}$, we see that $\bra{\sigma}\Delta^-_p\ket{\sigma} = | \del_p \ket{\sigma} |^2 \geq 0$, and since
        $\bra{\sigma}\Delta^-_p\ket{\sigma} = \lambda^-_\sigma\braket{\sigma}{\sigma}$, we also have that $\lambda-_\sigma \geq 0.$\\
        Furthermore, since $\Delta^+_p\Delta^-_p = \Delta^-_p\Delta^+_p = 0$ we have that $[\Delta^+_p,\Delta^-_p] = 0$, thence
        $[\Delta_p, \Delta^\pm_p] = 0$, therefore $\Delta_p, \Delta^+_p, \Delta^-_p$ share a basis of eigenchains.
        Let $\ket{\sigma}$ be in that common basis then $\Delta_p\ket{\sigma} = \lambda_\sigma\ket{\sigma}$, where
        $\lambda_\sigma = \lambda^+_\sigma + \lambda^-_\sigma \geq 0$. \qedhere
    \end{proof}

    Another really interesting property that was first proven by Beno Eckmann in 1944, is that the kernel of the p-Laplacian
    is isomorphic to the p-homology group.

    \begin{thm}
        Let $\Delta_p$ be a laplacian operator, then $ker\Delta_p \simeq H_p$.
    \end{thm}
    \begin{proof}
        We recall the definition of laplacian $\Delta_p =: \del_{p+1}\del^\dagger_{p+1} + \del^\dagger_p\del_p$.\\
        Because of the homology lemma $\Delta^+_p\Delta-_p = \Delta^-_p\Delta^+_p = 0$, therefore $ker\Delta^{\pm}_p \subset im\Delta^{\mp}_p.$\\
        It is trivial to see that, since $ker\Delta_p = ker\Delta^+_p \cap ker\Delta^-_p $, we have that $ker\del^\dagger_{p+1} \cap ker\del_p \subset ker\Delta_p.$ \\
        Less trivial is the opposite inclusion, in fact, let $\ket{\sigma} \in ker\Delta_p$, we have that $\del_{p+1}^\dagger\ket{\sigma} \in ker\del_{p+1} = (im\del_{p+1}^\dagger)^\perp$ 
        and of course that $\del_{p+1}^\dagger\ket{\sigma} \in im\del_{p+1}^\dagger$.\\
        Therefore, since our integration is an inner product, because of the adjunction relation we have that the chain $\del_{p+1}^\dagger\ket{\sigma}$ must
        be the null chain, hence the inclusion $\ker\Delta_p \subset \ker\del_{p+1}^\dagger$. Similarly we can see that $\ker\Delta_p \subset \ker\del_p$,        
        therefore $ker\Delta_p \subset ker\del^\dagger_{p+1} \cap ker\del_p $. Now, since $ker\Delta_p = ker\del^\dagger_{p+1} \cap ker\del_p$, we conclude that 
        $ker\Delta_p = (im\del_{p+1}^\dagger)^\perp \cap ker\del_p \simeq H_p$. \qedhere
    \end{proof}

    Let $\{\ket{e_i}\}_{i \in I}$ be the canonical basis of chains for $C_p$, namely the chains corresponding to the $p$-simplexes of the simplicial complex.
    Nevertheless can also choose the laplacian eigenchains $\{\ket{i}\}_{i \in I}$, such that $\Delta_p\ket{i} = \lambda_i\ket{i}$, as a basis for $C_p$. This change of basis
    is reflected on a change of coordinates called \ii{simplicial Fourier transform}.

    \begin{defn}
        Let $\ket{f} \in C_p$, where $dimC_p = n_p$, we define the \ii{simplicial Fourier transform} $\mc{F}_p : \R^{n_p} \to \R^{n_p}$ to be
        \[ (\braket{e_i}{f})_{i \in I} \mapsto (\braket{i}{f})_{i \in I},\]
        where $I = \{1,...,n_p\}$.
    \end{defn}

    This transform only defines a change of basis since $\ket{f} = \sum_{i \in I}\braket{e_i}{f}\ket{e_i} = \sum_{i \in I}\braket{i}{f}\ket{i}$, and therefore is invertible.
    We can in fact represent the simplicial Fourier transform with the matrix $F^{-1}_{ij} = F^\dagger_{ij} := \braket{i}{e_j}$, and its inverse $F_{ij} := \braket{e_i}{j}$.
    To define a convolution between two $p$-chains we use the famous convolution theorem $\mc{F}(f * \psi) = \mc{F}(f)\mc{F}(\psi)$. The laplacian is diagonalized by the simplicial
    Fourier transform, i.e. $\Delta_p = F diag(\lambda_i)_{i \in I} F^\dagger$.

    \begin{defn}
        Let $\{\ket{i}\}_{i \in I}$ be a basis such that $\Delta_p\ket{i} = \lambda_i\ket{i}$, let $\ket{f},\ket{\psi} \in C_p$, we define the representatives of $\ket{f * \psi}$ on
        the laplacian eigenchains to be 
        \[ \braket{i}{f * \psi} := \braket{i}{f}\braket{i}{\psi} \quad \forall i \in I.\]
        Therefore $\ket{f * \psi} = \sum_{i \in I}\braket{i}{f}\braket{i}{\psi}\ket{i}$.
    \end{defn}

    The filters used in \cite{simplicialNN} are low degree polynomials in the frequency domain, for instance a filter defined by $\braket{i}{\psi_\mu} = \sum_{n = 0}^N \mu_n \lambda_i^n$.
    This way one can easily define the convolution on the canonical basis as 
    \[\braket{e_i}{f * \psi_\mu} = \sum_{j \in I}\braket{e_i}{j}\braket{j}{f*\psi_\mu} = \sum_{n = 0}^N\mu_n \bra{e_i}(\sum_{j \in I}\ket{j}\lambda_j^n\bra{j})\ket{f}
     = \sum_{n=0}^N \mu_n \bra{e_i}\Delta_p^n\ket{f},\]
    therefore $\ket{f*\psi_\mu} = \sum_{n = 0}^N \mu_n \Delta_p^n\ket{f}$.
    

\end{document}
