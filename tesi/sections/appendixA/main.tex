\documentclass[../1.tex]{subfiles}
\begin{document}

    At the beginning of \autoref{ch:1} we mentioned that a relevant concept in algebriac topology is that of conversion form
    structures where a problem cannot be solved to structures where the problem can be solved. Not all conversions are good ones,
    we need the language of category theory to define the rules that make a conversion a good conversion, i.e. functoriality.

    \begin{defn}
        A \ii{category} $\bb{C}$ consists of three ingerdients:\\
        a class of \ii{objects} $Obj(\bb{C})$; sets of \ii{morphisms}
        $Hom(A,B)$ for every ordered pair $(A,B) \in Obj(\bb{C}) \times Obj(\bb{C})$; a composition $Hom(A,B) \times Hom(B,C) \to Hom(A,C)$, denoted
        by $(f,g) \mapsto f \circ g$ for every $A,B,C \in Obj(\bb{C})$, satisfying the following axioms:
        \begin{enumerate}[(i)]
            \item the family of $Hom(A,B)$ is pairwise disjoint,
            \item the composition, when defined, is associative,
            \item for each $A \in Obj(\bb{C})$ there exists an \ii{identity} $1_A \in Hom(A,A)$ such that for $f \in Hom(A,B)$ and $g \in Hom(C,A)$
                  we have that $1_A \circ f = f$ and $g \circ 1_A = g$.
        \end{enumerate}
    \end{defn}

    Instead of writing $f \in Hom(A,B)$, we usually write $f : A \to B$. 

    \begin{defn}
        Let $\bb{A}$ and $\bb{C}$ be categories, a \ii{functor} $T : \bb{A} \to \bb{C}$ is a function, that is, 
        \begin{enumerate}[(i)]
            \item $A \in Obj(\bb{A}) \implies TA \in Obj(\bb{C})$,
            \item $f:A \to A' \implies Tf: TA \to TA' \quad A,A' \in Obj(\bb{A})$,
            \item if $f,g$ are morphisms in $\bb{A}$ for which $g \circ f$ is defined, then $T(g \circ f) = (Tg) \circ (Tf)$,
            \item $T(1_A)= 1_{TA} \quad \forall A \in \bb{A}$.
        \end{enumerate}
    \end{defn}

    The property $(iii)$ of the previous definition actually defines what we shall call \ii{covariant functors}.
    If instead we require $T(g \circ f) = (Tf) \circ (Tg)$, we are defining a so called \ii{contravariant functor}.

    \begin{defn}
        An \ii{equivalence} in a category $\bb{C}$ is a morphism $f : A \to B$ for which there exists a morphism $g : B \to A$ such that
        $ f \circ g = 1_B$ and $ g \circ f = 1_A$.
    \end{defn}

    \begin{thm}
        If $\bb{A}$ and $\bb{C}$ are categories and $T : \bb{A} \to \bb{C}$ is a functor of either variance, then whenever $f$ is 
        an equivalence on $\bb{A}$ then $Tf$ is an equivalence on $\bb{C}$.
    \end{thm}
    \begin{proof}
        We apply $T$ to the equations $ f \circ g = 1_B$ and $ g \circ f = 1_A$, that for a covariant functor leads to 
        $(Tf) \circ (Tg) = T(1_B) = 1_{TB}$ and $(Tg) \circ (Tf) = T(1_A) = 1_{TA}$.
    \end{proof}

    A category that will be used in the following section is the category of topological spaces and continuous functions.

    \begin{prop}
        Topological spaces and continuous functions are a category \bb{Top}, whose equivalences are called homeomorphisms.
    \end{prop}

    Other examples of categories can be found in \cite{rotman}.

\end{document}