\documentclass[../1.tex]{subfiles}
\begin{document}

    In this appendix we review few notions of category theory.
    This theory is very useful to stduy homology and cohomology theory, however in this thesis we employ this language only marginally.
    We refer the reader to \cite{mclane} for a full account of category theory.

    \begin{defn}
        A \ii{category} $\bb{C}$ consists of three ingredients:
        \begin{enumerate}[1.]
            \item a class of \ii{objects} $Obj(\bb{C})$,
            \item sets of \ii{morphisms} $Hom(A,B)$ for every ordered pair $(A,B) \in Obj(\bb{C}) \times Obj(\bb{C})$,
            \item a composition $Hom(A,B) \times Hom(B,C) \to Hom(A,C)$, denoted by $(f,g) \mapsto f \circ g$ for every $A,B,C \in Obj(\bb{C})$, satisfying the following axioms:
            \begin{enumerate}[(i)]
                \item the family of $Hom(A,B)$ is pairwise disjoint,
                \item the composition, when defined, is associative,
                \item for each $A \in Obj(\bb{C})$ there exists an \ii{identity} $1_A \in Hom(A,A)$ such that for $f \in Hom(A,B)$ and $g \in Hom(C,A)$ we have that $1_A \circ f = f$ and $g \circ 1_A = g$.
            \end{enumerate}
        \end{enumerate}
    \end{defn}

    Instead of writing $f \in Hom(A,B)$, we usually write $f : A \to B$. 

    \begin{defn}
        Let $\bb{A}$ and $\bb{C}$ be categories, a \ii{functor} $T : \bb{A} \to \bb{C}$ is a function, that is, 
        \begin{enumerate}[(i)]
            \item for each $A \in Obj(\bb{A})$ it assigns $TA \in Obj(\bb{C})$,
            \item for each morphism $f:A \to A'$ it assigns a morphism $Tf: TA \to TA' \quad \forall A,A' \in Obj(\bb{A})$,
            \item if $f,g$ are morphisms in $\bb{A}$ for which $g \circ f$ is defined, then $T(g \circ f) = (Tg) \circ (Tf)$,
            \item $T(1_A)= 1_{TA} \quad \forall A \in \bb{A}$.
        \end{enumerate}
    \end{defn}

    The property (iii) of the previous definition actually defines what we shall call \ii{covariant functors}.
    If instead we require $T(g \circ f) = (Tf) \circ (Tg)$, we are defining a \ii{contravariant functor}.

    \begin{defn}
        An \ii{equivalence} in a category $\bb{C}$ is a morphism $f : A \to B$ for all $A,B \in Obj(\bb{C})$ for which there exists a morphism $g : B \to A$ such that
        $ f \circ g = 1_B$ and $ g \circ f = 1_A$.
    \end{defn}

    \begin{thm}
        If $\bb{A}$ and $\bb{C}$ are categories and $T : \bb{A} \to \bb{C}$ is a functor of either variance, then whenever $f$ is 
        an equivalence on $\bb{A}$ then $Tf$ is an equivalence on $\bb{C}$.
    \end{thm}
    \begin{proof}
        We apply $T$ to the equations $ f \circ g = 1_B$ and $ g \circ f = 1_A$, that for a covariant functor leads to 
        $(Tf) \circ (Tg) = T(1_B) = 1_{TB}$ and $(Tg) \circ (Tf) = T(1_A) = 1_{TA}$.
    \end{proof}

    A category that will be used in the following section is the category of topological spaces and continuous functions.

    \begin{prop}
        Topological spaces and continuous functions between them are a category \bb{Top}, whose equivalences are called homeomorphisms.
    \end{prop}

    {\color{blue}
    \begin{prop}
        Abelian groups and group homomorphisms between them are a category \bb{Ab}, whose equiavlences are called homomorphism.
    \end{prop}

    Other examples of categories can be found in \cite{rotman} at 0.3 and in \cite{mclane}.

    An important example of functor comes from the theory of homology.
    When we defined the homology groups we focused on simplicial complexes, however if a simplicial complex $K$ is a triangulation of a topological space $X$
    we can identify $H_i(X) := H_i(K)$. This is possible because withing the category \bb{Top} any two homeomorphic spaces are equivalent, and every merely topological consideration
    about a particular topological space is also valid for spaces homeomorphic to the one in object.

    \begin{thm}
        For $i \geq 0$, $H_i : \bb{Top} \to \bb{Ab}$ is a functor called \ii{homology functor}.
    \end{thm}

    We already know how the homology functor acts on the objects, the way it acts on morphisms will be excluded from this appendix and can be found in 
    \cite{rotman} at $4.3$. As we saw, we know that functors preserve the equivalence between the objects of the categories, therefore if two topological spaces are isomorphic
    their homology groups will be isomorphic.}
\end{document}