\documentclass[../1.tex]{subfiles}
\begin{document}

    \begin{defn}
        A \ii{category} $\bb{C}$ consists of three ingerdients:
        \begin{enumerate}[1.]
            \item a class of \ii{objects} $Obj(\bb{C})$,
            \item sets of \ii{morphisms} $Hom(A,B)$ for every ordered pair $(A,B) \in Obj(\bb{C}) \times Obj(\bb{C})$,
            \item a composition $Hom(A,B) \times Hom(B,C) \to Hom(A,C)$, denoted by $(f,g) \mapsto f \circ g$ for every $A,B,C \in Obj(\bb{C})$, satisfying the following axioms:
            \begin{enumerate}[(i)]
                \item the family of $Hom(A,B)$ is pairwise disjoint,
                \item the composition, when defined, is associative,
                \item for each $A \in Obj(\bb{C})$ there exists an \ii{identity} $1_A \in Hom(A,A)$ such that for $f \in Hom(A,B)$ and $g \in Hom(C,A)$
                      we have that $1_A \circ f = f$ and $g \circ 1_A = g$.
            \end{enumerate}
        \end{enumerate}
    \end{defn}

    Instead of writing $f \in Hom(A,B)$, we usually write $f : A \to B$. 

    \begin{defn}
        Let $\bb{A}$ and $\bb{C}$ be categories, a \ii{functor} $T : \bb{A} \to \bb{C}$ is a function, that is, 
        \begin{enumerate}[(i)]
            \item for each $A \in Obj(\bb{A})$ it assigns $TA \in Obj(\bb{C})$,
            \item for each morphism $f:A \to A'$ it assigns a morphism $Tf: TA \to TA' \quad \forall A,A' \in Obj(\bb{A})$,
            \item if $f,g$ are morphisms in $\bb{A}$ for which $g \circ f$ is defined, then $T(g \circ f) = (Tg) \circ (Tf)$,
            \item $T(1_A)= 1_{TA} \quad \forall A \in \bb{A}$.
        \end{enumerate}
    \end{defn}

    The property (iii) of the previous definition actually defines what we shall call \ii{covariant functors}.
    If instead we require $T(g \circ f) = (Tf) \circ (Tg)$, we are defining a \ii{contravariant functor}.

    \begin{defn}
        An \ii{equivalence} in a category $\bb{C}$ is a morphism $f : A \to B$ for all $A,B \in Obj(\bb{C})$ for which there exists a morphism $g : B \to A$ such that
        $ f \circ g = 1_B$ and $ g \circ f = 1_A$.
    \end{defn}

    \begin{thm}
        If $\bb{A}$ and $\bb{C}$ are categories and $T : \bb{A} \to \bb{C}$ is a functor of either variance, then whenever $f$ is 
        an equivalence on $\bb{A}$ then $Tf$ is an equivalence on $\bb{C}$.
    \end{thm}
    \begin{proof}
        We apply $T$ to the equations $ f \circ g = 1_B$ and $ g \circ f = 1_A$, that for a covariant functor leads to 
        $(Tf) \circ (Tg) = T(1_B) = 1_{TB}$ and $(Tg) \circ (Tf) = T(1_A) = 1_{TA}$.
    \end{proof}

    A category that will be used in the following section is the category of topological spaces and continuous functions.

    \begin{prop}
        Topological spaces and continuous functions are a category \bb{Top}, whose equivalences are called homeomorphisms.
    \end{prop}

    Other examples of categories can be found in \cite{rotman} at 0.3 and in \cite{mclane}.

\end{document}