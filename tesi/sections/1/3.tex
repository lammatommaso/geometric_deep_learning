\documentclass[../1.tex]{subfiles}
\begin{document}

    In order to define data on simplicial complexes we are interested in studying the dual of the chains $\ket{\sigma}$
    which we shall call cochains $\bra{\sigma}$.
       
    \begin{defn}
        Let $C_p$ be the group p-chains, we define the group of \ii{p-cochains} to 
        be $C^p := Hom(C_p, G)$, i.e. the dual space of $C_p$.
    \end{defn}

    \begin{defn}
        The dual of the boundary maps which we shall call
        \ii{coboundary maps}, is the group homomorphsm defined by
        \[ d_{p+1} : C^p \to C^{p+1} \quad d_p(\bra{\sigma}) := \bra{\sigma} \circ \del_{p+1} \forall \bra{\sigma} \in C^p.\]
    \end{defn}

    One could also show that the property $d^p+1 \circ d_p$ is verified.
    
    \begin{defn}
        We define the \ii{p-cohomology group} to be 
        \[H^p := \frac{ker d_{p+1}}{im d_{p}},\] 
        where $im d_{p-1}$ is the group of simplicial p-cocycles and
        $ker d_p$ is the group of simplicial p-coboundaries.
    \end{defn}

    To have a mirrored intuition of what the coboundary operator actually does we equipe our simplicial chain spaces with
    an inner product. This finite dimensional Hilbert space structure shall allow us to represent the cochain $\bra{\sigma}$ evaluated on the chain $\ket{\tau}$ 
    as the scalar product $\braket{\sigma}{\tau}$.

    \begin{defn}
        We define the scalar product called \ii{integration} $\braket{}{} : C_p \times C_p \to \R$ on the canonical basis of $C_p$ to be 
        \[ \braket{i}{j} = \delta_{ij} ,\]  
        where $i,j$ are any two p-simplexes in the canonical basis of $C_p$ and $\delta_{ij}$ is the Kronecker Delta.
    \end{defn}

    \hfill \\
    The same Hilbert space structure allows us to represent the coboundary operator $d$ in the space of chains with its integral kernel $\del^\dagger$.

    \begin{defn}
        Let $d_{p+1} : C^p \to C^{p+1}$ be the coboundary operator, then for any $\bra{\sigma} \in C^p, \tau \in C_{p+1}$ we can define its dual representation $\del^\dagger_{p+1}$ by
        \[d_{p+1} \bra{\sigma})\ket{\tau} = \braket{\del^\dagger_{p+1} \sigma}{\tau}. \]
    \end{defn}

    It is easy to notice that our definitions lead to the restatement of the equivalent of the generalized Stokes' theorem on simplicial complexes
    according to the integration previously defined, i.e. $(d\bra{\sigma})\ket{\tau} = \braket{\sigma}{\del\tau}$. One can therefore think of the coboundary
    operator as a discrete exterior derivative acting on cochains.

\end{document}
