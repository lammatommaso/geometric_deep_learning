\documentclass[../1.tex]{subfiles}

\begin{document}

    Homology and cohomology are key concepts in algebraic topology. We shall discuss homology theory to the extent
    that allows us to define the laplacian operator on simplicial complexes, for supplementary readings see \cite{singerthorpe} at 6.1 or \cite{hatcher} at 2.1.
    First we want to equip our simplicial complexes with an orientation. So far we have considered the simplex $\{x_i\}_{i \in I}$ up to reorderings of the index set $I$, but in most applications this is not the case.\\
   
    We denote by $\{\sigma_i\}_{i \in I}$, where $I = \{1,...,n_p\}$, the set of all $p$-simplexes of the complex each one with a choice
    of orientation.

    \begin{defn}
        \label{chaingroup}
        Let $\mc{A}$ be a simplicial complex, we define the \ii{group of p-chains} 
        \[ \mc{C}_p(\mc{A},\R) := \{ \sum_{i \in I} \lambda_i \{\sigma_i\} : \lambda_i \in \R \}.\]
    \end{defn}

    In the definition\autoref{chaingroup} the sum $\sum_{i \in I} \lambda_i \{\sigma_i\}$ is the formal sum of the free abelian group on $\R$, see \cite{lang} at 1.7.
    This can be generalized from $\R$ to an arbitrary abelian group, nevertheless, for the most of the applications, the groups $\Z,\R,\Z_2$ are considered. 

    \begin{defn}
        \label{orientedchains}
        Let $\mc{A}$ be a simplicial complex, we define the \ii{group of oriented p-chains} as
        the group $\mc{C}_p(\mc{A},\R)$ subject to $ \{x_i\}_{i \in I} = sgn(\pi)\{x_i\}_{i \in \pi(I)}$, where $\pi:I \to I$ is 
        any permutation of the index set. We shall denote this group as $C_p(\mc{A},\R)$, and its chains with the notation $\ch{x_i}_{i \in I}$.
    \end{defn}

    To keep our notation light we shall write $C_p$ instead of $C_p(\mc{A},\R)$. The constraints due to the 
    permutations in definition\autoref{orientedchains} simply implies that whenever we swap two vertexes in the chain
    we must multiply by $(-1)$.
    
    \begin{prop}
        $C_p$ is an abelian group.
    \end{prop}

    A particularly relevant role in homology theory is played by the \ii{boundary map}. First we define the boundary of an oriented simplex.

    \begin{defn}
        Let $\ch{\sigma} = \ch{x_0,...,x_{p+1}}$ be an oriented $(p+1)$-simplex. The boundary $\del\ch{\sigma}$ of $\ch{\sigma}$ is the $p$-chain
        defined by \[ \del_{p+1}\ch{\sigma} := \sum_{i = 0}^{p+1}(-1)^i\ch{x_0,...,\widehat{x_i},...,x_{p+1}}\]
        where the  $\widehat{\color{white} a}$  over a symbol means that symbol is deleted.
    \end{defn}

    \begin{rem}
        Note that whenever we are able to construct a geometric realization for the oriented simplicial complex, the set $\bigcup_{i = 0}^{p+1}[x_0,...,\widehat{x_i},...,x_{p+1}]$
        is the topological boundary of $[\sigma]$.
    \end{rem}

    Furthermore we are able to extend the boundary from simplexes to chains.
    
    \begin{defn}
        We define the \ii{boundary map} $\del_{p+1} : C_{p+1} \to C_p$ to be the group homomorphism defined by
        \[ \del_{p+1}(\sum_{i \in I}\lambda_i\ch{\sigma_i}) := \sum_{i \in I}\lambda_i\del_{p+1}\ch{\sigma_i} .\]
    \end{defn}

    An important property of boundary maps is the following, which we shall denote as \ii{homology lemma}.

    \begin{lem}
        The boundary maps satisfy $\del_p \circ \del_{p+1} = 0$.
    \end{lem}
    \begin{proof}
        Since the boundary maps are linear it is sufficient to check this on the generators.
        Let $\del_{p+1}\ch{x_0,\dots,x_{p+1}} = \sum_{i=0}^{p+1} (-1)^i\ch{x_0,\dots,\widehat{x_i},\dots,x_{p+1}}$ then 
        \[(\del_p \circ \del_{p+1})\ch{x_0,\dots,x_{p+1}} = \sum_{j=0, j \neq i}^{p+1} \sum_{i=0}^{p+1} (-1)^{i+j}\ch{x_0,\dots,\widehat{x_i},\dots,\widehat{x_j},\dots,x_{p+1}} = 0. \qedhere \]     
    \end{proof}

    The homology lemma is necessary to define the \ii{homology group}.
    The homology group is , intuitively,  the space of cycles that are not boundaries. In fact, without the
    homology lemma the quotient would not well defined since $im\del_{p+1} \subset ker\del_p$ would not be satisfied.

    \begin{defn}
        We define the \ii{p-homology group} to be 
        \[H_p := \frac{\ker\del_p}{\im\del_{p+1}},\] 
        where $\im\del_{p+1}$ is called the group of simplicial \ii{p-cycles} and
        $\ker\del_p$ is called the group of simplicial \ii{p-boundaries}.
    \end{defn}

\end{document}
