\documentclass[../1.tex]{subfiles}
\begin{document}

    In order to define data on simplicial complexes we are interested in studying the dual of the simplicial chain complex
    which we shall call a  simplicial cochain complex.
    % The cochain complex is the image of the chain complex via a special functor,
    % i.e. the dual contravariant functor.

    % \begin{defn}
    %     Let \bb{V} be the category of vector spaces and linear transformations between them, we define the \ii{dual
    %     contravariant functor} by
    %     \[ V \in Obj(\bb{V}) \mapsto V^* := Hom(V,\R) \in Obj(\bb{V}) \]
    %     \[ \del \in Hom(U,V) \mapsto \del^* \in Hom(V^*,U^*) : \del^*(\phi)=\phi \circ \del . \]
    % \end{defn}

    % By applying this functor to the spaces of simplicial chains we are able to define the space of simplicial cochains, 
    % and their respective coboundary operator.
    
    \begin{defn}
        Let $C_p$ be the space of simplicial p-chains, we define the space of \ii{simplicial cochains} to 
        be $C^p := Hom(C_p, \R)$, i.e. the dual space of $C_p$.
    \end{defn}

    \begin{defn}
        The dual of the boundary operator which we shall call
        \ii{coboundary operator}, is defined to be 
        \[ d_{p+1} : C^p \to C^{p+1} \quad d_p(\sigma^*) = \sigma^* \circ \del_{p+1} .\]
    \end{defn}

    One could also show that the property $d^2 = 0$ is verified in the so defined cochain complex.
    The homology group could in general be define for any chain complex, in particulare for the simplicial cochain
    complex it takes the name of cohomology group.

    \begin{defn}
        We define the \ii{p-cohomology group} to be 
        \[H^p := \frac{ker d_{p+1}}{im d_{p}},\] 
        where $im d_{p-1}$ is the group of simplicial p-cocycles and
        $ker d_p$ is the group of simplicial p-coboundaries.
    \end{defn}

    To have a mirrored intuition of what the coboundary operator actually does we equipe our simplicial chain spaces with
    an inner product. This finite dimensional Hilbert space structure shall allow us to represent the action of a cochain $\sigma^*$ on the chain $\tau$ 
    as the scalar product $\scal{\sigma}{\tau}$.

    \begin{defn}
        We define the scalar product called \ii{integration} $\langle, \rangle : C_p \times C_p \to \R$ on the canonical basis of $C_p$ to be 
        \[ \langle i, j \rangle = \delta_{ij} ,\]  
        where $i,j$ are any two p-simplexes in the canonical basis of $C_p$ and $\delta_{ij}$ is the Kronecker Delta.
    \end{defn}

    It is convenient, from now on,  to write cochains $\sigma^*$ as bra vectors $\bra{\sigma}$, chains $\sigma$ as ket vectors $\ket{\sigma}$,
    and the scalar product $\scal{i}{j}$ as the bra-ket product $\braket{i}{j}$. The $\delta_{ij}$ can always be replaced with a positive definite
    weight matrix $w^p_{ij}$.\\
    \hfill \\
    The same Hilbert space structure allows us to represent the coboundary operator $d$ in the space of chains with its integral kernel $\del^\dagger$.

    \begin{defn}
        Let $d_{p+1} : C^p \to C^{p+1}$ be the coboundary operator, then for any $\sigma^* \in C^p, \tau \in C_{p+1}$ we can define its dual representation $\del^\dagger_{p+1}$ by
        \[ (d_{p+1} \sigma^*)(\tau) = (d_{p+1} \bra{\sigma})\ket{\tau} = \braket{\del^\dagger_{p+1} \sigma}{\tau}. \]
    \end{defn}

    It is easy to notice that our definitions lead to the restatement of the equivalent of the generalized Stokes' theorem on simplicial complexes
    according to the integration previously defined, i.e. $(d\bra{\sigma})\ket{\tau} = \braket{\sigma}{\del\tau}$. One can therefore think of the coboundary
    operator as a discrete exterior derivative acting on cochains.
   
    % \begin{thm}
    %     $H_p \simeq H^p$
    % \end{thm}
    % \begin{proof}
    %     Let $A_p, A_{p+1}$ be the real matrices associated to the boundary operators $\del_p, \del_{p+1}$ on the canonical basis of p-simplexes,
    %     then we have that since $\langle \sigma, A_{p+1} \tau \rangle = \langle \sigma, A^\dagger_{p+1} \tau \rangle$ also have that
    %     \[ H_p =: \frac{ker\del_p}{im\del_{p+1}} = \frac{kerA_p}{imA_{p+1}},\]
    %     \[ H^p =: \frac{kerd_{p+1}}{imd_p} = \]
    %     Let's now write the dimensional equations for $A_p, A_{p+1}$,
    %     \[ n_{p+1} := dim(C_{p+1}) = dim(imA_{p+1}) + dim(kerA_{p+1}) \]
    %     \[ n_p := dim(C_p) = dim(imA_p) + dim(kerA_p) \]
    %     \[ dim(H_p) = n_p - dim(imA_p) - (n_{p+1}-dim(kerA_{p+1})) \]
    % \end{proof}

\end{document}
