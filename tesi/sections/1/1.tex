\documentclass[../1.tex]{subfiles}
\begin{document}

    \begin{defn}
        Let $\mc{V}$ be a finite set we define an \bb{abstract simplicial complex} $\mc{A}$ to be 
        \[\mc{A}:=\{ \sigma  \subset \mc{V} : \tau \subset \sigma \Rightarrow \tau \in \mc{A} \}\] 
        where $\sigma$ are called \bb{abstract simplexes} of $\mc{A}$.
    \end{defn}
    
    One calls $\mc{V}$ the \bb{vertex set} of $\mc{A}$ and denotes it by $Vert(\mc{A})$; since the vertex
    set is finite we expect every abstract simplex to be finite, therefore we might use the notation $\sigma = \{ v_i \}_{i \in I_\sigma}$,
    which so far we consider invariant under arbitrary permutations on the index set $I_\sigma$.

    \begin{defn}
        Let $\mc{A}$ be an abstract simplicial complex we define its \bb{dimension} to be
        \[ dim \mc{A} := max_{\sigma \in \mc{A}}(|\sigma|-1), \]
        where $|\sigma|$ is the cardinality of $\sigma$.
    \end{defn}

    One calls an abstract simplex of dimension $p$ an \bb{abstract p-simplex}, according to our
    definition the empy set is a $(-1)$-simplex. A \bb{graph} is a one dimensional abstract simplicial complex.

    \begin{defn}
        Let $\mc{A},\mc{B}$ be abstract simplicial complexes, then a \bb{simplicial map} $\phi : \mc{A} \to \mc{B}$ is a function 
        such that whenever $\sigma = \{v_i\}_{i \in I_\sigma} \in \mc{A}$, then $\phi(\{v_i\}_{i \in I_\sigma}) = \{\phi(v_i)\}_{i \in I_\sigma} \in \mc{B}$,
        where $\phi(v_i) \in Vert(\mc{B}) \, \forall i \in I_\sigma$.
    \end{defn}

    Although the vertex to vertex mapping is a quite selective condition on the function we did not prevent it from cramming
    abstract simplexes into lower dimensional ones.

    \begin{thm}
        All abstract simplicial complexes and simplicial maps are a category $\mf{A}$ whose identities are called isomorphisms.
    \end{thm}

    Although abstract simplicial complex can be used to model any kind of vertex interaction they lack of a topology, we wish
    therefore to define some structures in a euclidean space that can be related unequivocally(i.e. via a functor) to abstract simplicial complexes.
    We shall call those geometric simplicial complexes to avoid misunderstandings.
        
    \begin{defn}
        Let $I$ be a finite set of indexes, we define the \bb{convex envelope} of the points $\{x_i\}_{i \in I} \subset \R^n$ to be 
        \[ \langle x_i \rangle_{i \in I} := \{ a = \sum_{i \in I} \lambda_i x_i : \lambda_i \in \R,\ \lambda_i > 0,\  \sum_{i \in I} \lambda_i = 1 \},\]
        which is the smallest convex set containing $\{x_i\}_{i \in I}$.
    \end{defn}

    \begin{defn}
        Let $\{x_i\}_{i \in I} \subset \R^n$ we define the points $\{x_i\}_{i \in I}$ to be \bb{affinely independent} if and only if
        \[ \sum_{i \in I} \lambda_i x_i = \sum_{i \in I} \mu_i x_i \quad \Rightarrow \quad 
        \lambda_i = \mu_i \; \forall i \in I ,\]
        where $\sum_{i \in I} \lambda_i = \sum_{i \in I} \mu_i = 1$.
    \end{defn}

    \begin{defn}
        We define a \bb{geometric p-simplex} to be a convex envelop $\langle x_i \rangle_{i \in I}$
        where $\{x_i\}_{i \in I} \subset \R^n$ are affinely independent and $|I| = p + 1$.
    \end{defn}
    
    \begin{defn}
        Let $\sigma$ be a geometric p-simplex, we say that another t-simplex $\tau$ is a \bb{face} of $\sigma$ or equivalently 
        that $\sigma$ is a \bb{coface} of $\tau$, by our notiation $\tau \leq \sigma$, if and only if $\tau \subset \sigma$, where $t \leq p$.
    \end{defn}

    \begin{defn}
        We define a \bb{geometric simplicial complex} $\mc{G}$ to be a collection of geometric simplexes such that
        \begin{enumerate}[(i)]
            \item $ \tau \leq \sigma \in \mc{G} \Rightarrow \tau \in \mc{G} $,
            \item $ \sigma, \tau \in \mc{G} \Rightarrow \sigma \cap \tau \in \mc{G}  $.
        \end{enumerate}
    \end{defn}

    \newpage

    \begin{defn}
        \bb{Geometric realization of an abstract simplicial complex}\\
        Let $\mc{K}$ be a geometric simplicial complex, and let $Vert(\mc{K}) := \{\sigma \in \mc{K} : dim(\sigma) = 0\}$, we call
        the abstract simplicial complex $\mc{A} := \{ \{x_i\}_{i \in I} \subset Vert(\mc{K}) : \langle x_i \rangle_{i \in I} \in \mc{K} \}$
        a vertex scheme for $\mc{K}$ or equivalently we might say that $\mc{K}$ is a geometric realization of $\mc{A}$.
    \end{defn}
    \begin{thm}
        Let $\mc{A}$ be a d-dimentional abstract simplicial complex, it admits a geometric realization in $\R^{2d+1}$.
    \end{thm}
    Kuratowski theorem proves the prevuois statement to be also sharp.
\end{document}
