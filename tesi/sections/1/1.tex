\documentclass[../1.tex]{subfiles}
\begin{document}
    \begin{defn}
        \bb{Abstract simplicial complex (finite)}\\
	Let $\mc{F}$ be a family of sets we then define an abstract simplicial complex $\mc{A}$ to be 
	\[\mc{A}:=\{ \sigma = \{A_i\}_{i \in I_\sigma} \subset \mc{F}: \tau \subset \sigma \Rightarrow \tau \in \mc{A} \}\] 
	where $I_\sigma$ is a finite set of indexes, we shall call $\sigma$ abstract simplexes of $\mc{A}$.
    \end{defn}
    \begin{defn}
        \bb{Dimension of an abstract simplicial complex}\\
	Let $\mc{A} = \{\sigma_j\}_{j \in J}$ be an abstract simplicial complex we define its dimension to be
	\[ dim \mc{A} := max_{\sigma_j \in \mc{A}}dim(\sigma_j), \]
	where $dim(\sigma_j) := |\sigma_j|-1$.
    \end{defn}
    \begin{defn}
        \bb{Abstract graph}\\
	An abstract graph $\mc{G} = \{\sigma_j\}_{j \in J}$ is a 1-dimensional abstract simplicial complex whose 
	vertexes and edges are respectively
	\[ \mc{V} := \{ \sigma_j \in \mc{G} : dim(\sigma_j) = 0\} \ and \]
	\[ \mc{E} := \{ \sigma_j \in \mc{G} : dim(\sigma_j) = 1\} \ .\]
    \end{defn}
    In Definition 1.1.1. we tacitly assumed the definition of the abstract simplex $\sigma_j$ invariant
    with respect to permutations of the indexes $I_j$, this assumption establishes the difference between
    directed and undirected graphs.
    \begin{defn}
        \bb{Convex envelop of points in $\R^n$}\\
        Let $\{x_i\}_{i \in I} \subset \R^n$ we define the convex envelope of $\{x_i\}_{i \in I}$ to be 
        \[ \langle x_i \rangle_{i \in I} := \{ a = \sum_{i \in I} \lambda_i x_i : \lambda_i \in \R,\ \lambda_i > 0,\  \sum_{i \in I} \lambda_i = 1 \},\]
        which is the smallest convex set containing $\{x_i\}_{i \in I}$.
    \end{defn}
    \begin{defn}
        \bb{Affine independency of points in $\R^n$}\\
    \end{defn}
    \begin{defn}
        \bb{Geometric k-simplexes}\\
    \end{defn}
    \begin{defn}
        \bb{Geometric Simplicial Complex}\\
    \end{defn}
    \begin{thm}
        \bb{Geometric realization of an abstract simplicial complex}\\
    \end{thm}
\end{document}
