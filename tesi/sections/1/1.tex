\documentclass[../1.tex]{subfiles}
\begin{document}

    In this section we shall define structures called simplicial complexes and discuss some of their properties. In order to define these structures
    we need the definitions of convex hull and affine independence in $\R^n$. For a deeper insight into simplicial complexes we refer to \cite{singerthorpe,rotman}.    

    \begin{defn}
        Let $A \subset \R^n$, we define $A$ to be \ii{convex} if
        \[ x,y \in A \then tx + (1+t)y \in A \]
        for all $t \in [0,1]$.
    \end{defn}

    \begin{defn}
        Let $\sigma := \{x_i\}_{i \in I}$ be a finite subset of $\R^n$, we define $\sigma$ to be \ii{affinely independent} if 
        $\{ x_0 - x_i\}_{i \in I \m \{0\}}$ is linearly independent.
    \end{defn}

    For the definition of affine independence of $\sigma = \{x_i\}_{i \in I} \subset \R^n$ to be well stated we need it to be independent of the choice of $x_0$.
      
    \begin{prop}
        Let $\sigma := \{x_i\}_{i \in I}$ be a finite subset of $\R^n$, let $j \in I$ then, if $\{ x_j - x_i\}_{i \in I \m \{j\}}$ is linearly independent,
        also $\{ x_0 - x_i\}_{i \in I \m \{0\}}$ is. 
    \end{prop}
    \begin{proof}
        If $j=0$ the statement is trivially true. Let $j \neq 0$ and $\lambda_i \in \R$ for all $i \neq j$, then
        \[ \sum_{i \in I \m \{j\}} \lambda_i (x_j - x_i) = 0 \then \lambda_i = 0 \quad \forall i \in I-\{j\}.\]
        Let then $\mu_i \in \R$ for all $i \neq 0$, and suppose 
        \[ \sum_{i \in I \m \{0\}} \mu_i (x_0 - x_i) = (x_0 - x_j) \sum_{i \in I \m \{0\}} \mu_i + \sum_{i \in I \m \{0\}} \mu_i (x_j - x_i) = 0.\]
        If we define $\mu_0 := -\sum_{i \in I \m \{0\}} \mu_i$ we have that
        \[ 0 = \sum_{i \in I} \mu_i (x_j - x_i) = \sum_{i \in I \m \{j\}} \mu_i (x_j - x_i) \then \mu_i = 0 \quad \forall i \in I \m \{j\},\]
        therefore the definition of affine independence is well stated.
    \end{proof}

    \begin{defn}
        Let $\sigma := \{x_i\}_{i \in I}$ be a finite subset of $\R^n$, we define the \ii{covex set generated by $\sigma$} to be 
        the smallest convex set containing $X$ according to the inclusion relation.
    \end{defn}

    Since the intersection of convex sets is convex, the convex set generated by $\sigma$ can be equivalently defined as the intersection 
    of all convex sets containing $\sigma$.

    \begin{thm}
        Let $\sigma := \{x_i\}_{i \in I}$ be a finite subset of $\R^n$, if $X$ is affinely independent then the convex set generated by $X$ is 
        \[ [\sigma] := \{ \sum_{i \in I} \lambda_i x_i : \lambda_i \geq 0, \sum_{i \in I} \lambda_i = 1\}. \]
        Furthermore for any point $x \in [\sigma]$ we have that
        \[ x = \sum_{i \in I} \lambda_i x_i = \sum_{i \in I} \mu_i x_i \then \lambda_i = \mu_i \forall i \in I, \] 
        where $\lambda_i,\mu_i \geq 0$ and $\sum_{i \in I} \lambda_i = \sum_{i \in I} \mu_i =1$.
        \label{thm:1}
    \end{thm}

    \begin{proof}
        Let $C := \{ \bigcap_\alpha C_\alpha : \sigma \in C_\alpha, C_\alpha$ convex$\}$, we divide the proof in three steps:
        \begin{enumerate}[(i)]
            \item $C \subset [\sigma]$.\\
            This is true if $[\sigma]$ is convex and contains $\sigma$.
            The proof that it contains $\sigma$ is trivial. In fact for every vertex $x_j = \sum_{i \in I} \delta_{ij} x_i$,
            and $\sum_{i \in I}\delta_{ij} = 1$.\\
            To prove that it is convex we chose two points $a = \sum_{i \in I} a_i x_i, b = \sum_{i \in I} b_i x_i$ where $a_i,b_i \geq 0 \forall i \in I$
            and $\sum_{i \in I} a_i = \sum_{i \in I} b_i = 1$. For $t \in [0,1]$
            \[ ta+(1-t)b = t\sum_{i \in I} a_i x_i + (1-t) \sum_{i \in I} b_i x_i = \sum_{i \in I} (ta_i + (1-t)b_i)x_i.\]
            It follows that $ta_i + (1-t)b_i \geq 0$ and $\sum_{i \in I} (ta_i + (1-t)b_i) = t\sum_{i \in I}a_i + (1-t)\sum_{i \in I}b_i = 1$ for all $i \in I$,
            which proves our statement.

            \item $[\sigma] \subset C$.\\
            If all but one the $\lambda_i$ are zero certaintly $\sum_{i \in I} \lambda_i x_i \in C$, since $C$ contains all the vertexes.
            The inuctive hypothesis, by relabeling, is that if the first $\lambda_0,...,\lambda_{n-1}$ are non-zero, hence not even $1$, then $\sum_{i \in I} \lambda_i x_i \in C$.
            We want to show that whenever $\lambda_0,...,\lambda_n$ are non-zero then also $\sum_{i \in I} \lambda_i x_i \in C$, since $\lambda_n \neq 0$ we have that
            \[ \sum_{i \in I} \lambda_i x_i =  \sum_{i = 0}^n \lambda_i x_i = \lambda_n x_n + \sum_{i = 0}^{n-1} \lambda_i x_i =
               \lambda_n x_n + (1 - \lambda_n) \sum_{i = 0}^{n-1} \frac{\lambda_i}{1 - \lambda_n} x_i.\]
            Since $\sum_{i = 0}^{n-1} \frac{\lambda_i}{1 - \lambda_n} = 1$, for the inductive hypothesis $\sum_{i = 0}^{n-1} \frac{\lambda_i}{1 - \lambda_n} x_i \in C$.
            Also the vertex $x_n$ is contained in $C$ by definition, therefore, being $C$ convex and $\lambda_n \in [0,1]$, it follows that 
            \[ \lambda_n x_n + (1 - \lambda_n) \sum_{i = 0}^{n-1} \frac{\lambda_i}{1 - \lambda_n} x_i \in C. \]
            Accordingly $\sum_{i \in I} \lambda_i x_i \in C$, by induction we conclude the proof.

            \item $\sum_{i \in I} \lambda_i x_i = \sum_{i \in I} \mu_i x_i \then \lambda_i = \mu_i \forall i \in I$.\\
            Let $\sum_{i \in I} \lambda_i x_i = \sum_{i \in I} \mu_i x_i$, then also $x_0\sum_{i \in I} \lambda_i + \sum_{i \in I} \lambda_i(x_i-x_0) = 
            x_0\sum_{i \in I} \mu_i +\sum_{i \in I} \mu_i (x_i - x_0)$, and since both $\lambda_i$ and $\mu_i$ are normalised we have that
            \[ \sum_{i \in I} (\lambda_i - \mu_i) (x_0 - x_i) = \sum_{i \in I \m \{0\}} (\lambda_i - \mu_i) (x_0 - x_i) = 0 \then \lambda_i = \mu_i \quad \forall i \in I \m \{0\},\]
            because of the affine independence. \qedhere
        \end{enumerate}
    \end{proof}

    The set $[\sigma]$ is often called \ii{covex hull} of $\sigma$.
    
    % \begin{exa}
    %     Let ${A,B,C,D} \subset \R^n$ representing the four vertxes of a square and let $A,D$ be opposite vertxes, one can easily see that 
    %     \[ D = A + (B-A) + (C-A) = (-1)A + (1)B + (1)C + (0)D = (0)A + (0)B + (0)C + (1)D \, ,\]
    %     therefore $A,B,C,D$ are not affinely independent.
    %     In fact the $4$ vertexes of a square are coplanar, i.e. they belong to the same $2$ dimensional affine space.
    % \end{exa}

    % image

    \begin{defn}
        We define a \ii{p-simplex} $[\sigma]$ to be the convex hull of an affinely independent set $\sigma := \{x_i\}_{i \in I} \subset \R^n$,
        where $p = |I|-1$ is called dimension of the p-simplex. 
    \end{defn}

    Theorem \ref{thm:1} gives us the possibility to represent a point in a simplex $[\sigma]$ via a finite set of real parameters defined in the range $[0,1]$
    and satisfying the normalisation condition $\sum_{i \in I } \lambda_i = 1$. Such parameters are called \ii{baricentric coordinates} of $[\sigma]$.\\
    \hfill \\
    The points in $\sigma$ are called \ii{vertexes} of the simplex $[\sigma]$, accordingly we define the vertex set of a simplex $[\sigma]$ to be 
    $Vert([\sigma]) = \sigma$.
    
    \begin{defn}
        Let $[\sigma]$ be a p-simplex and $p,t \in \N$, we say that another t-simplex $[\tau]$ is a \ii{face} of $[\sigma]$ or equivalently 
        that $[\sigma]$ is a \ii{coface} of $[\tau]$, and we write $[\tau] \leq [\sigma$], if $\tau \subset \sigma$, where $t \leq p$.
    \end{defn}

    Now we are ready for our main definitions.
    
    \begin{defn}
        We define a \ii{simplicial complex} $\mc{G}$ to be a collection of simplexes such that
        \begin{enumerate}[(i)]
            \item if any simplex $ [\tau] \leq [\sigma]$ and $[\sigma] \in \mc{G}$, then $ [\tau] \in \mc{G}$,
            \item if $ [\sigma], [\tau] \in \mc{G}$, then $[\sigma] \cap [\tau] \in \mc{G}$.
        \end{enumerate}
    \end{defn}

    
    \begin{figure}[h]
        \centering
        \includegraphics[width=4cm, height=4cm]{sections/1/complex}
        \caption{Example of simplicial complex.}
        \label{fig:1}
    \end{figure} 

    \begin{figure}[h]
        \centering
        \includegraphics[width=3cm, height=5cm]{sections/1/noncomplex}
        \caption{Set of simplexes which is not a simplicial complex.}
        \label{fig:2}
    \end{figure}

    In fact, in \autoref{fig:2} that the intersection property of simplicial complexes is not satisfied.\\
    \hfill \\
    % Simplicial complexes are the objects of our category, we now look for appropriate morphisms.

    % \begin{defn}
    %     Let $\mc{G},\mc{H}$ be simplicial complexes, then a \ii{simplicial map} $\phi : \mc{G} \to \mc{H}$ is a function 
    %     such that whenever $[ x_i ]_{i \in I} \in \mc{G}$, then $\phi([ x_i ]_{i \in I}) = [ \phi(v_i) ]_{i \in I} \in \mc{H}$,
    %     where $\phi(x_i) \in Vert(\mc{H}) \, \forall i \in I$.
    % \end{defn}

    % \begin{thm}
    %     Simplicial complexes and simplicial maps are a category $\bb{G}$, whose equivalences are called isomorphisms..
    % \end{thm}

    % \begin{defn}
    %     Let $\mc{G}$ be a simplicial complex, we define its \ii{underlying space} $|\mc{G}| = \bigcup_{\sigma \in \mc{G}} \sigma$, provided with
    %     the standard topology inherited from $\R^n$.
    % \end{defn}

    % Since the union of compact sets is compact the underlying space of a simplicial complex in $\R^n$ is a compact topological subspace of $\R^n$.

    % \begin{defn}
    %     A topological space $X$ is called \ii{polyhedron} if there exists a simplicial complex $\mc{G}$ and a homeomorphism
    %     $h : |\mc{G}| \to X$. The ordered pair $(\mc{G}, h)$ is called a \ii{triangulation} of $X$.
    % \end{defn}

    % One understands that in order to have a homeomorphism between the compact underlying space of a simplcial complex and another topological
    % space, this other space has to be compact.

    % \begin{lem}
    %     Let a topological space $X$ be a finite union of closed subsets $X = \bigcup_{i \in I} X_i$. If, for some space Y, there are continuous
    %     maps $f_i : X_i \to Y$ that agree on overlaps $f_i |_{X_i \cap X_j} = f_j |_{X_i \cap X_j}$, there there exist a unique continuous
    %     function $f : X \to Y$ such that $f |_{X_i} = f_i \, \forall i \in I$.
    % \end{lem}

    % \begin{defn}
    %     Let $\phi : \mc{G} \to \mc{H}$ be a simplicial map, let then $\sigma \in \mc{G}$ we define $f_\sigma : \sigma \to |\mc{H}|$ to be 
    %     $\sum_{v \in Vert(\sigma)} \lambda_v v \mapsto \sum_{v \in Vert(\sigma)} \lambda_v \phi(v)$. The continuity of this functions in $\sigma$ and the
    %     intersection property of the definition of simplicial complex allow us to use the previous lemma to uniquely define a function $|\phi| : |\mc{G}| \to |\mc{H}|$
    %     which we shall name \ii{piecewise linear map}.
    % \end{defn}

    % The unique association of simplicial complexes and their underlying spaces and of simlicial maps and piecewise linear maps leads to the definition
    % of a functor from the category of simplicial complexes and maps to the category of topological spaces and continuous functions.

    % \begin{thm}
    %     $|\;| : \bb{G} \to \bb{Top}$ is a functor.
    % \end{thm}

    % Notice that there is no obvious functor from $\bb{Top}$ to $\bb{G}$, therefore the implications reguarding equivalences are strictly directed.\\

    \newpage
    In this first part we analysed simplicial complexes as peculiar subsets of $\R^n$, namely \ii{geometric simplicial complexes}.
    Although this approach provides simplicial complexes with the topology inherited from the metric space, it hides the power of simplicial complexes 
    to describe those networks and interactions which would happily exist without that topology. To make this distinction clear enough we will treat
    simplicial complexes as a realization of more abstract objects called \ii{abstract simplicial complexes}. A richer discussion of abstract simplicial complexes
    can be found in \cite{rotman,comptop}.
    
    \begin{defn}
        Let $\mc{V}$ be a finite set, we define an \ii{abstract simplicial complex} $\mc{A}$ to be 
        a family of non empy subsets of $\mc{V}$ such that:
        \begin{enumerate}[(i)]
            \item if $v \in \mc{V}$, then $\{v\} \in \mc{A},$
            \item if $\sigma \in \mc{A}$ and $\tau \subset \sigma$, then $\tau \in \mc{A}.$
        \end{enumerate}
        We call the member of this family \ii{abstract simplexes}.
    \end{defn}
    
    One calls $\mc{V}$ the \ii{vertex set} of $\mc{A}$ and denotes it by $Vert(\mc{A})$; since the vertex
    set is finite we expect every abstract simplex to be also finite, therefore we might use the notation $\sigma = \{ v_i \}_{i \in I_\sigma}$,
    which so far we consider invariant under arbitrary permutations of the finite index set $I_\sigma$.
    
    \begin{defn}
        Let $\mc{A}$ be an abstract simplicial complex and $\mc{G}$ a geometric simplicial complex, if for all $\{x_i\}_{i \in I} \in \mc{A}$ also $[x_i]_{i \in I} \in \mc{G}$
        we say that $\mc{G}$ is a \ii{geometric realization} of $\mc{A}$.
    \end{defn}

    While every geometric simplicial complex can be thought as a geometric realization of an abstract simplicial complex, the existence of a geometric 
    realization for an arbitrary abstract simplicial complex is not trivial at all.

    \begin{thm}
        \label{thm:2}
        Let $\mc{A}$ be an $n$-dimensional abstract simplicial complex, then it admits a geometric realization in $\R^{2n+1}$.
    \end{thm}

    A proof of this theorem can be found in \cite{comptop}.\\
    \hfill \\
    Both for abstract and geometric simplicial complexes one can define maps called \ii{simplicial maps} in order to obtain a category whose
    equivalences are called isomorphisms. The geometric realisation is unluckily not a functor on the whole category of abstract simplicial complexes due to the
    limitation imposed by \autoref{thm:2}. Nevertheless, when they exist, the geometric realizations of isomorphic abstract simplicial complexes are theirselves isomorphic.
    A short discussion of category theory can be found in Appendix \ref{app:A}.\\
    \hfill \\
    In the following sections we shall use abstract simplicial complexes, which can be always thought geometrically in the appropriate euclidean space.
    % Furthermore, the \ii{simplicial approximation theorem}, which can be found in \cite{singerthorpe,rotman,comptop}, allows us to define a functor from the category of 
    % simplicial complexes and maps to that of topological spaces and contnuous functions. The preimage of a topological space via this functor is called its \ii{triangulation}.
    % \hfill \\
    % Similarly the triangulations of homeomorphic topological spaces are isomorphic.



    % \begin{defn}
    %     Let $\mc{A},\mc{B}$ be abstract simplicial complexes, then a \ii{simplicial map} $\phi : \mc{A} \to \mc{B}$ is a function 
    %     such that whenever $\sigma = \{v_i\}_{i \in I_\sigma} \in \mc{A}$, then $\phi(\{v_i\}_{i \in I_\sigma}) = \{\phi(v_i)\}_{i \in I_\sigma} \in \mc{B}$,
    %     where $\phi(v_i) \in Vert(\mc{B}) \, \forall i \in I_\sigma$.
    % \end{defn}

    % Although the vertex to vertex mapping is a quite selective condition on the function we did not prevent it from cramming
    % abstract simplexes into lower dimensional ones.

    % \begin{thm}
    %     Abstract simplicial complexes and simplicial maps are a category $\bb{A}$, whose equivalences are called isomorphisms.
    % \end{thm}

    % One can show that under some dimensional conditions one can define a functor from the category of abstract simplicial complexes
    % to the category of simplicial complexes.

    % \begin{defn}
    %     Let $\mc{G}$ be a simplicial complex,  we call the abstract simplicial complex 
    %     \[ \mc{A} := \{ \{x_i\}_{i \in I} \subset Vert(\mc{G}) : [x_i]_{i \in I} \in \mc{G} \} \]
    %     a vertex scheme for $\mc{G}$ or equivalently we might say that $\mc{G}$ is a \ii{geometric realization} of $\mc{A}$.
    % \end{defn}

    % \begin{thm}
    %     Let $\mc{A}$ be a d-dimensional abstract simplicial complex, it admits a geometric realization in $\R^{2d+1}$.
    % \end{thm}
    % Kuratowski theorem proves the prevuois statement to be also sharp.\\
    % \hfill \\
    % One could also show that that geometric realizations of isomorphic abstract simplicial complexes are themselves isomorphic.
\end{document}
