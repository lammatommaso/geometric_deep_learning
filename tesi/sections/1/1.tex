\documentclass[../1.tex]{subfiles}
\begin{document}

    The essential idea in algebraic topology is to convert problems about topological spaces and continuous functions into
    problems about algebraic objects and their homomorphisms, this way one hopes to end up with an easier problem to solve. 
    The language of category theory shall be our main tool to formally describe this conversion.

    \begin{defn}
        A \ii{category} $\bb{C}$ consists of three ingerdients:\\
        a class of \ii{objects} $obj\bb{C}$; sets of \ii{morphisms}
        $Hom(A,B)$ for every ordered pair $(A,B) \in obj\bb{C}$; a composition $Hom(A,B) \times Hom(B,C) \to Hom(A,C)$, denoted
        by $(f,g) \mapsto f \circ g$ for every $A,B,C \in obj\bb{C}$, satisfying the following axioms:
        \begin{enumerate}[(i)]
            \item the family of $Hom(A,B)$ is pairwise disjoint,
            \item the composition, when defined, is associative,
            \item for each $A \in obj\bb{C}$ there exists an \ii{identity} $1_A \in Hom(A,A)$ such that for $f \in Hom(A,B)$ and $g \in Hom(C,A)$
                  we have that $1_A \circ f = f$ and $g \circ 1_A = g$.
        \end{enumerate}
    \end{defn}

    \begin{thm}
        Topological spaces and continuous functions are a category \bb{Top}, whose equivalences are called homeomorphisms.
    \end{thm}

    Instead of writing $f \in Hom(A,B)$, we usually write $f : A \to B$. For other examples of categories see \cite{rotman}.

    \begin{defn}
        Let $\bb{A}$ and $\bb{C}$ be categories, a \ii{functor} $T : \bb{A} \to \bb{C}$ is a function, that is, 
        \begin{enumerate}[(i)]
            \item $A \in obj\bb{A} \implies TA \in obj\bb{C}$,
            \item $f:A \to A' \implies Tf: TA \to TA' \quad A,A' \in obj\bb{A}$,
            \item if $f,g$ are morphisms in $\bb{A}$ for which $g \circ f$ is defined, then $T(g \circ f) = (Tg) \circ (Tf)$,
            \item $T(1_A)= 1_{TA} \quad \forall A \in \bb{A}$.
        \end{enumerate}
    \end{defn}

    The property $(iii)$ of the previous definition actually defines what we shall call \ii{covariant functors}.
    If instead we require $T(g \circ f) = (Tf) \circ (Tg)$, we are defining a so called \ii{contravariant functor}.

    \begin{defn}
        An \ii{equivalence} in a category $\bb{C}$ is a morphism $f : A \to B$ for which there exists a morphism $g : B \to A$ such that
        $ f \circ g = 1_B$ and $ g \circ f = 1_A$.
    \end{defn}

    \begin{thm}
        If $\bb{A}$ and $\bb{C}$ are categories and $T : \bb{A} \to \bb{C}$ is a functor of either variance, then whenever $f$ is 
        an equivalence on $\bb{A}$ then $Tf$ is an equivalence on $\bb{C}$.
    \end{thm}
    \begin{proof}
        We apply $T$ to the equations $ f \circ g = 1_B$ and $ g \circ f = 1_A$, that for a covariant functor leads to 
        $(Tf) \circ (Tg) = T(1_B) = 1_{TB}$ and $(Tg) \circ (Tf) = T(1_A) = 1_{TA}$.
    \end{proof}

\end{document}