\documentclass[../1.tex]{subfiles}
\begin{document}

    We shall now introduce algebraic objects called simplicial complexes and see how they are related to compact
    topological spaces. In order to do that we require the definitions of convex envelope and affine independence of points in $\R^n$.

    \begin{defn}
        Let $I$ be a finite set of indexes, we define the \ii{convex envelope} of the points $\{x_i\}_{i \in I} \subset \R^n$ to be 
        \[ [ x_i ]_{i \in I} := \{ \sum_{i \in I} \lambda_i x_i : \lambda_i \in \R,\ \lambda_i \geq 0,\  \sum_{i \in I} \lambda_i = 1 \}.\]
    \end{defn}

    It is easy to see that convex envelopes are convex and compact sets with respect to the standard topology in $\R^n$.
    From now, if not otherwise specified, we shall assume $I$ to be a finite set of indexes.

    \begin{prop}
        Let $\{x_i\}_{i \in I} \subset \R^n$ then $[ x_i ]_{i \in I}$ is the smallest convex set containing $X$.
    \end{prop}

    The order by which we define the smallest convex set is the one given by the relation $\subseteq$.

    % Possible example with some image

    \begin{defn}
        Let $\{x_i\}_{i \in I} \subset \R^n$ we define the points $\{x_i\}_{i \in I}$ to be \ii{affinely independent} if and only if
        \[ \sum_{i \in I} \lambda_i x_i = \sum_{i \in I} \mu_i x_i \quad \Rightarrow \quad 
        \lambda_i = \mu_i \; \forall i \in I ,\]
        whenever $\sum_{i \in I} \lambda_i = \sum_{i \in I} \mu_i = 1$.
    \end{defn}

    For a more intuitive understanding of this definition one can say that $n$ points in some euclidean space are affinely independent if and only if they do
    not belong to the same $n-2$ dimensional affine space.

    \begin{exa}
        Let ${A,B,C,D} \subset \R^n$ representing the four vertxes of a square and let $A,D$ be opposite vertxes, one can easily see that 
        \[ D = A + (B-A) + (C-A) = (-1)A + (1)B + (1)C + (0)D = (0)A + (0)B + (0)C + (1)D \, ,\]
        therefore $A,B,C,D$ are not affinely independent.
        In fact the $4$ vertexes of a square are coplanar, i.e. they belong to the same $2$ dimensional affine space.
    \end{exa}

    \begin{defn}
        We define a \ii{p-simplex} to be a convex envelop $[ x_i ]_{i \in I}$
        where $\{x_i\}_{i \in I} \subset \R^n$ are affinely independent and $|I| = p + 1$, where $|I|$ is the cardinality
        of $I$.
    \end{defn}
    
    One denotes the vertex set $\{ x_i \}_{i \in I}$ of a simplex $\sigma = [ x_i ]_{i \in I}$ by $Vert(\sigma)$.
    
    \begin{defn}
        Let $\sigma$ be a $p-simplex$, we say that another t-simplex $\tau$ is a \ii{face} of $\sigma$ or equivalently 
        that $\sigma$ is a \ii{coface} of $\tau$, by our notiation $\tau \leq \sigma$, if and only if $\tau \subset \sigma$, where $t \leq p$.
    \end{defn}

    Simplexes can therefore be points, segments, triangles, tetrahedra or higher dimensional sets which I cannot name, if these particularly simple
    sets can describe topological spaces we can stop complicating things and try to define a category of simplexes. Unfortunately convex spaces are
    not able to describe topological spaces with holes.

    \begin{defn}
        We define a \ii{simplicial complex} $\mc{G}$ to be a collection of simplexes such that
        \begin{enumerate}[(i)]
            \item $ \tau \leq \sigma \in \mc{G} \Rightarrow \tau \in \mc{G}$,
            \item $ \sigma, \tau \in \mc{G} \Rightarrow \sigma \cap \tau \in \mc{G}$.
        \end{enumerate}
    \end{defn}

    Let $A,B,C$ be vertexes of a triangle, with the simplicial complex $\{[A],[B],[C],[A,B],[B,C],[C,A]\}$ one can describe the boundary of a triangle, which has
    a hole that could not be described by any simplex.\\
    \hfill \\
    Simplicial complexes are the objects of our category, we now look for appropriate morphisms.

    \begin{defn}
        Let $\mc{G},\mc{H}$ be simplicial complexes, then a \ii{simplicial map} $\phi : \mc{G} \to \mc{H}$ is a function 
        such that whenever $[ x_i ]_{i \in I} \in \mc{G}$, then $\phi([ x_i ]_{i \in I}) = [ \phi(v_i) ]_{i \in I} \in \mc{H}$,
        where $\phi(x_i) \in Vert(\mc{H}) \, \forall i \in I$.
    \end{defn}

    \begin{thm}
        Simplicial complexes and simplicial maps are a category $\bb{G}$, whose equivalences are called isomorphisms..
    \end{thm}

    \begin{defn}
        Let $\mc{G}$ be a simplicial complex, we define its \ii{underlying space} $|\mc{G}| = \bigcup_{\sigma \in \mc{G}} \sigma$, provided with
        the standard topology inherited from $\R^n$.
    \end{defn}

    Since the union of compact sets is compact the underlying space of a simplicial complex in $\R^n$ is a compact topological subspace of $\R^n$.

    \begin{defn}
        A topological space $X$ is called \ii{polyhedron} if there exists a simplicial complex $\mc{G}$ and a homeomorphism
        $h : |\mc{G}| \to X$. The ordered pair $(\mc{G}, h)$ is called a \ii{triangulation} of $X$.
    \end{defn}

    One understands that in order to have a homeomorphism between the compact underlying space of a simplcial complex and another topological
    space, this other space has to be compact.

    \begin{lem}
        Let a topological space $X$ be a finite union of closed subsets $X = \bigcup_{i \in I} X_i$. If, for some space Y, there are continuous
        maps $f_i : X_i \to Y$ that agree on overlaps $f_i |_{X_i \cap X_j} = f_j |_{X_i \cap X_j}$, there there exist a unique continuous
        function $f : X \to Y$ such that $f |_{X_i} = f_i \, \forall i \in I$.
    \end{lem}

    \begin{defn}
        Let $\phi : \mc{G} \to \mc{H}$ be a simplicial map, let then $\sigma \in \mc{G}$ we define $f_\sigma : \sigma \to |\mc{H}|$ to be 
        $\sum_{v \in Vert(\sigma)} \lambda_v v \mapsto \sum_{v \in Vert(\sigma)} \lambda_v \phi(v)$. The continuity of this functions in $\sigma$ and the
        intersection property of the definition of simplicial complex allow us to use the previous lemma to uniquely define a function $|\phi| : |\mc{G}| \to |\mc{H}|$
        which we shall name \ii{piecewise linear map}.
    \end{defn}

    The unique association of simplicial complexes and their underlying spaces and of simlicial maps and piecewise linear maps leads to the definition
    of a functor from the category of simplicial complexes and maps to the category of topological spaces and continuous functions.

    \begin{thm}
        $|\;| : \bb{G} \to \bb{Top}$ is a functor.
    \end{thm}

    Notice that there is no obvious functor from $\bb{Top}$ to $\bb{G}$, therefore the implications reguarding equivalences are strictly directed.\\
    \hfill \\
    Although this approach provides simplicial complexes with the topology inherited from the metric space it hides the power of simplicial complexes 
    to describe those networks and interactions which would happily exist without that topology, to make this distinction clear enough we will treat
    simplicial complexes as a realization of more abstract objects called abstract simplicial complexes.
    
    \begin{defn}
        Let $\mc{V}$ be a finite set we define an \ii{abstract simplicial complex} $\mc{A}$ to be 
        \[\mc{A}:=\{ \sigma  \subset \mc{V} : \tau \subset \sigma \Rightarrow \tau \in \mc{A} \}\] 
        where $\sigma$ are called \ii{abstract simplexes} of $\mc{A}$.
    \end{defn}
    
    One calls $\mc{V}$ the \ii{vertex set} of $\mc{A}$ and denotes it by $Vert(\mc{A})$; since the vertex
    set is finite we expect every abstract simplex to be also finite, therefore we might use the notation $\sigma = \{ v_i \}_{i \in I_\sigma}$,
    which so far we consider invariant under arbitrary permutations of the finite index set $I_\sigma$.

    \begin{defn}
        Let $\mc{A}$ be an abstract simplicial complex we define its \ii{dimension} to be
        \[ dim \mc{A} := max_{\sigma \in \mc{A}}(|\sigma|-1), \]
        where by $|\sigma|$ we denote the cardinality of $\sigma$.
    \end{defn}

    One calls an abstract simplex of dimension $p$ an \ii{abstract p-simplex}, according to our
    definition the empy set is a $(-1)$-simplex. A \ii{graph} is a one dimensional abstract simplicial complex.

    \begin{defn}
        Let $\mc{A},\mc{B}$ be abstract simplicial complexes, then a \ii{simplicial map} $\phi : \mc{A} \to \mc{B}$ is a function 
        such that whenever $\sigma = \{v_i\}_{i \in I_\sigma} \in \mc{A}$, then $\phi(\{v_i\}_{i \in I_\sigma}) = \{\phi(v_i)\}_{i \in I_\sigma} \in \mc{B}$,
        where $\phi(v_i) \in Vert(\mc{B}) \, \forall i \in I_\sigma$.
    \end{defn}

    Although the vertex to vertex mapping is a quite selective condition on the function we did not prevent it from cramming
    abstract simplexes into lower dimensional ones.

    \begin{thm}
        Abstract simplicial complexes and simplicial maps are a category $\bb{A}$, whose equivalences are called isomorphisms.
    \end{thm}

    One can show that under some dimensional conditions one can define a functor from the category of abstract simplicial complexes
    to the category of simplicial complexes.

    \begin{defn}
        Let $\mc{G}$ be a simplicial complex,  we call the abstract simplicial complex 
        \[ \mc{A} := \{ \{x_i\}_{i \in I} \subset Vert(\mc{G}) : [x_i]_{i \in I} \in \mc{G} \} \]
        a vertex scheme for $\mc{G}$ or equivalently we might say that $\mc{G}$ is a \ii{geometric realization} of $\mc{A}$.
    \end{defn}

    \begin{thm}
        Let $\mc{A}$ be a d-dimentional abstract simplicial complex, it admits a geometric realization in $\R^{2d+1}$.
    \end{thm}
    Kuratowski theorem proves the prevuois statement to be also sharp.\\
    \hfill \\
    One could also show that that geometric realizations of isomorphic abstract simplicial complexes are themselves isomorphic.
\end{document}
