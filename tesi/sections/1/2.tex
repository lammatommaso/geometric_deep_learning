\documentclass[../1.tex]{subfiles}

\begin{document}
    \begin{defn}
        \bb{Linear space of simplicial p-chains}\\
        Let $\mc{A}$ be an abstract simplicial complex, and let $\mc{A}_p := \{\sigma \in \mc{A} : dim(\sigma) = p\}$, 
        we define the linear space $C_p = C_p(\mc{A})$ of simplicial p-chain on $\mc{A}$ to be 
        \[ C_p = \{ \sum_{\sigma \in \mc{A}_p} \lambda^\sigma \sigma ,\quad \lambda^\sigma \in \Z_2\}, \]
        where the formal operations of the linear space are given by the defitnition itself.
    \end{defn}
    The set $\mc{A}^p$ is a canonical base of p-simplexes for $C_p$.\\
    \begin{defn}
        \bb{Boundary operator on $C_{p+1}$}\\
        Let $\sigma$ be an element of the canonical base of $C_{p+1}$ we define $\del : C_{p+1} \to C_p$ such that
        \[ \del \sigma = \sum_{i = 0}^{p+1} (-1)^i \sigma_i, \]
        where if $ \sigma = \{x_0, ..., x_{p+1}\} \in C_{p+1}$ we define $\sigma_i := \{x_0, ..., x_{i-1},x_{i+1}, ..., x_{p+1} \in C_p\}$.\\
        Furthermore we extend this operator linearly on the whole space $C_{p+1}$
        \[ \del \left( \sum_{\sigma \in \mc{A}_p} \lambda^\sigma \sigma \right) = \sum_{\sigma \in \mc{A}_p} \lambda^\sigma \del \sigma \ .\]
    \end{defn}
    \begin{lem}
        Let $\sigma \in \mc{A}_{p+2}$, $p \geq 0$ then $\del^2 \sigma = 0$.
    \end{lem}
    \begin{proof}
        We have \\
        \begin{equation*}
            \begin{aligned}
                (\del^2 \sigma)_x &= \sum_{i = 0}^{p+2} (-1)^i (\del \sigma)_i \\
                &= \sum_{i = 0}^{p+2} (-1)^i \left[ \sum_{j = 0}^{i-1}(-1)^j \sigma_{ij} + 
                \sum_{j = i+1}^{p+2} (-1)^{j-1} \sigma_{ij} \right]\\
                &= \sum_{i = 0}^{p+2} \sum_{j = 0}^{i-1} (-1)^{i+j} \sigma_{ij} -
                \sum_{i = 0}^{p+2} \sum_{j = i+1}^{p+2} (-1)^{i+j} \sigma_{ij} = 0.
            \end{aligned}
        \end{equation*}
    \end{proof}
    \begin{defn}
        \bb{p-forms on abstract simplicial complexes}\\
        Let $\mc{A}$ be an abstract simplical complex we define the linear space of p-forms $\Lambda^p = \Lambda^p(\mc{A})$ to be 
        \[ \Lambda^p := \{ \omega : C_{p+1} \to \R\}, such \ that\]
        \[ \omega \left( \sum_{\sigma \in \mc{A}_{p+1}} \lambda^\sigma \sigma \right) = \sum_{\sigma \in \mc{A}_{p+1}} \lambda^\sigma \omega(\sigma)
        \quad \forall \omega \in \Lambda^p , \ \lambda_\sigma \in \Z_2.\]
        With the linear space operations defined as
        \[ + : \Lambda^p \times \Lambda^p \to \Lambda^p \quad
        \quad (\omega + \eta)(\sigma) = \omega(\sigma) + \eta(\sigma) \quad \sigma \in C_{p+1},\ \omega,\eta \in \Lambda^p, \]
        \[ \cdot : \R \times \Lambda^p \to \Lambda^p \quad
        \quad (\lambda \omega)(\sigma) = \lambda \omega(\sigma) \quad \sigma \in C_{p+1},\ \omega \in \Lambda^p,\ \lambda \in \R, \]
        where $dim(\Lambda^p) = |Vert(\mc{A})|^{p+1}$.
    \end{defn} 
    \begin{prop}
        A canonical base of elementary forms for $\Lambda^p$ is
        \[ \{e^\sigma \in \Lambda^p : \sigma \in \mc{A}_{p+1},\ e^\sigma(\tau) = \delta_{\sigma \tau} \quad \tau \in \mc{A}_{p+1}\},\]
        therefore giving us an expression for every other p-form
        \[ \omega = \sum_{\sigma \in \mc{A}_{p+1}} \omega_\sigma e^\sigma, \quad \omega_\sigma \in \R. \]
    \end{prop}
    \begin{defn}
        \bb{Exterior derivative of a p-form}\\
        Let $\omega \in \Lambda^p$ we define $d : \Lambda^p \to \Lambda^{p+1}$ on its coordinates to be 
        \[ (d \omega)_\sigma = \sum_{i = 0}^{p+1} (-1)^i \omega_{\sigma_i} \ . \]
    \end{defn}
    \begin{lem}
        Let $\omega \in \Lambda^p$, $p \geq 0$ then $d^2 \omega = 0$.
    \end{lem}
    \begin{proof}
        We have for $\sigma \in C_{p+3}$\\
        \begin{equation*}
            \begin{aligned}
                (d^2 \omega)_\sigma &= \sum_{i = 0}^{p+2} (-1)^i (d \omega)_{\sigma_i} \\
                &= \sum_{i = 0}^{p+2} (-1)^i \left[ \sum_{j = 0}^{i-1}(-1)^j \omega_{\sigma_{ij}} + 
                \sum_{j = i+1}^{p+2} (-1)^{j-1} \omega_{\sigma_{ij}} \right]\\
                &= \sum_{i = 0}^{p+2} \sum_{j = 0}^{i-1} (-1)^{i+j} \omega_{\sigma_{ij}} -
                \sum_{i = 0}^{p+2} \sum_{j = i+1}^{p+2} (-1)^{i+j} \omega_{\sigma_{ij}} = 0.
            \end{aligned}
        \end{equation*}
    \end{proof}
\end{document}
