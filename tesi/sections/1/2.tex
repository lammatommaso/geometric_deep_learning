\documentclass[../1.tex]{subfiles}
\begin{document}
    In \autoref{section:1.1} we studied simplicial complexes as subsets of $\R^n$. From now we shall call them \ii{geometric simplicial complexes}.
    Although this approach provides simplicial complexes with the topology inherited from the metric space, it hides the power of simplicial complexes 
    to describe networks and interactions which exist independently of that topology. To make this distinction clear, we will treat
    simplicial complexes as a realization of more abstract objects called \ii{abstract simplicial complexes}. A richer discussion of abstract simplicial complexes
    can be found in \cite{comptop} at 3.1 or in \cite{rotman} at 7.3.
    
    \begin{defn}
        Let $\mc{V}$ be a finite set, we define an \ii{abstract simplicial complex} $\mc{A}$ to be 
        a family of non empy subsets of $\mc{V}$ such that:
        \begin{enumerate}[(i)]
            \item if $v \in \mc{V}$, then $\{v\} \in \mc{A},$
            \item if $\sigma \in \mc{A}$ and $\tau \subset \sigma$, then $\tau \in \mc{A}.$
        \end{enumerate}
        We call the members of this family \ii{abstract simplexes}.
    \end{defn}
    
    We call $\mc{V}$ the \ii{vertex set} of $\mc{A}$ and denote it by $Vert(\mc{A})$; since the vertex
    set is finite, every abstract simplex is finite, therefore we can use the notation $\sigma = \{ v_i \}_{i \in I}$, to denote a simplex in $\mc{A}$.
    
    \begin{defn}
        Let $\mc{A}$ be an abstract simplicial complex and $\mc{G}$ a geometric simplicial complex, if for all $\{x_i\}_{i \in I} \in \mc{A}$ also $[x_i]_{i \in I} \in \mc{G}$
        we say that $\mc{G}$ is a \ii{geometric realization} of $\mc{A}$.
    \end{defn}

    While every geometric simplicial complex can be thought as a geometric realization of an abstract simplicial complex, the existence of a geometric 
    realization for an arbitrary abstract simplicial complex is not trivial at all.

    \begin{thm}
        \label{thm:2}
        Let $\mc{A}$ be an $n$-dimensional abstract simplicial complex, then it admits a geometric realization in $\R^{2n+1}$.
    \end{thm}

    A proof of this theorem can be found in \cite{comptop} at 3.1.\\
    \hfill \\
    Both for abstract and geometric simplicial complexes one can define maps called \ii{simplicial maps}. We obtain a category whose
    equivalences are called isomorphisms. 
    % The geometric realisation is unluckily not a functor on the whole category of abstract simplicial complexes due to the
    % limitation imposed by \autoref{thm:2}. Nevertheless, when they exist, the geometric realizations of isomorphic abstract simplicial complexes are theirselves isomorphic.
    A short discussion of category theory can be found in Appendix \autoref{app:A}.\\
    \hfill \\
    In the following sections we shall use abstract simplicial complexes, which can be always thought geometrically in the appropriate $\R^{2d+1}$.
    % Furthermore, the \ii{simplicial approximation theorem}, which can be found in \cite{singerthorpe,rotman,comptop}, allows us to define a functor from the category of 
    % simplicial complexes and maps to that of topological spaces and contnuous functions. The preimage of a topological space via this functor is called its \ii{triangulation}.
    % \hfill \\
    % Similarly the triangulations of homeomorphic topological spaces are isomorphic.



    % \begin{defn}
    %     Let $\mc{A},\mc{B}$ be abstract simplicial complexes, then a \ii{simplicial map} $\phi : \mc{A} \to \mc{B}$ is a function 
    %     such that whenever $\sigma = \{v_i\}_{i \in I_\sigma} \in \mc{A}$, then $\phi(\{v_i\}_{i \in I_\sigma}) = \{\phi(v_i)\}_{i \in I_\sigma} \in \mc{B}$,
    %     where $\phi(v_i) \in Vert(\mc{B}) \, \forall i \in I_\sigma$.
    % \end{defn}

    % Although the vertex to vertex mapping is a quite selective condition on the function we did not prevent it from cramming
    % abstract simplexes into lower dimensional ones.

    % \begin{thm}
    %     Abstract simplicial complexes and simplicial maps are a category $\bb{A}$, whose equivalences are called isomorphisms.
    % \end{thm}

    % One can show that under some dimensional conditions one can define a functor from the category of abstract simplicial complexes
    % to the category of simplicial complexes.

    % \begin{defn}
    %     Let $\mc{G}$ be a simplicial complex,  we call the abstract simplicial complex 
    %     \[ \mc{A} := \{ \{x_i\}_{i \in I} \subset Vert(\mc{G}) : [x_i]_{i \in I} \in \mc{G} \} \]
    %     a vertex scheme for $\mc{G}$ or equivalently we might say that $\mc{G}$ is a \ii{geometric realization} of $\mc{A}$.
    % \end{defn}

    % \begin{thm}
    %     Let $\mc{A}$ be a d-dimensional abstract simplicial complex, it admits a geometric realization in $\R^{2d+1}$.
    % \end{thm}
    % Kuratowski theorem proves the prevuois statement to be also sharp.\\
    % \hfill \\
    % One could also show that that geometric realizations of isomorphic abstract simplicial complexes are themselves isomorphic.
\end{document}
