\documentclass[../1.tex]{subfiles}
\begin{document}

    In order to define data on simplicial complexes we are interested in studying the dual of the chains $\ch{\sigma}$,
    which we shall call cochains $\varphi_{\ch{\sigma}}$.
       
    \begin{defn}
        Let $C_p$ be the group p-chains, we define the group of \ii{p-cochains} to 
        be $C^p := Hom(C_p, \R)$, i.e. the dual space of $C_p$.
    \end{defn}

    \begin{prop}
        The homomorphims $\{\varphi_{\ch{\sigma_i}}:C_p \to \R\}$ such that $\varphi_{\ch{\sigma_i}}(\ch{\sigma_j}) = \delta_{ij}$,
        form a basis of $C^p$.
    \end{prop}

    As well as with for chains, we have also for cochains a sequence of homeomorphisms called \ii{coboundary maps}. The coboundary maps are defined
    to be the dual of the boundary maps, hence satisfying a dual version of the homology lemma called \ii{cohomology lemma}.

    \begin{defn}
        The dual of the boundary maps which we shall call
        \ii{coboundary maps}, is the group homomorphsm defined by
        \[ d_{p+1} : C^p \to C^{p+1} \quad d_{p+1}\varphi_{\ch{\sigma}} := \varphi_{\ch{\sigma}} \circ \del_{p+1} \quad \forall \varphi_{\ch{\sigma}} \in C^p.\]
        Therefore $d_{p+1}\varphi_{\ch{\sigma}}(\ch{\tau}) = \varphi_{\ch{\sigma}} (\del_{p+1}\ch{\tau}) \quad \forall \ch{\tau} \in C_p$.
    \end{defn}

    The proof of $d_{p+1} \circ d_p = 0$ follow directly from the homology lemma. The cohomology lemma allows us to define the \ii{cohomology group}.
    
    \begin{defn}
        We define the \ii{p-cohomology group} to be 
        \[H^p := \frac{ker d_{p+1}}{im d_{p}},\] 
        where $im d_{p-1}$ is the group of simplicial p-cocycles and
        $ker d_p$ is the group of simplicial p-coboundaries.
    \end{defn}

    % To have a specular intuition of what the coboundary operator actually does, we equipe our simplicial chain groups with
    % an inner product. This finite dimensional Hilbert space structure shall allow us to represent the cochain $\varphi_{\ch{\sigma}}$ evaluated on the chain $\ch{\tau}$ 
    % as the inner product $\scal{\ch{\sigma}}{\ch{\tau}}$.

    % \begin{defn}
    %     We define the scalar product $\scal{\cdot}{\cdot} : C_p \times C_p \to \R$ on the canonical basis of $C_p$ to be 
    %     \[ \scal{\ch{\sigma_i}}{\ch{\sigma_j}} = \delta_{ij} ,\]  
    %     where $i,j$ are any two p-simplexes in the canonical basis of $C_p$ and $\delta$ is the Kronecker Delta.
    % \end{defn}

    % Another way to see this scalar product is as an integration.

    % \begin{defn}
    %     We define $\int^p:C_p\times C^p\to\R$ to be an \ii{integration} such that
    %     \[ \int^{p}_{\ch{\tau}}\varphi_{\ch{\sigma}} := \varphi_{\ch{\sigma}}(\ch{\tau}).\]
    % \end{defn}

    % From this point of view it is clear that the definition of coboundary is equivalent to a discrete generalized Stokes' theorem.

    % \begin{prop}
    %     From the definitions $\int^{p}_{\ch{\del_{p+1}\tau}}\varphi_{\ch{\sigma}}=\int^{p+1}_{\ch{\tau}}d_{p+1}\varphi_{\ch{\sigma}}$.
    % \end{prop}

    % One can therefore think of the coboundary operator as a discrete exterior derivative acting on cochains. 
    % Furthermore the choice of a measure of integration is equivalent to a choice of a basis for the chain group.

    % The same Hilbert space structure allows us to represent the coboundary operator $d$ acting on a cochain with its cochain representative.

    % \begin{defn}
    %     Let $d_{p+1} : C^p \to C^{p+1}$ be the coboundary operator, then for any \\ $\varphi_{\ch{\sigma}} \in C^p, \ch{\tau} \in C_{p+1}$ we can define its dual representation $\del^\dagger_{p+1}:C_p \to C_{p+1}$ by
    %     \[d_{p+1}\varphi_{\ch{\sigma}}(\ch{\tau}) = \varphi_{\del^\dagger_{p+1}\ch{\sigma}},\]
    %     where $\del_{p+1}^\dagger$ is the adjoint of $\del_{p+1}$ w.r.t. $\scal{\cdot}{\cdot}$, 
    %     namely $\scal{\ch{\sigma}}{\del_{p+1}\ch{\tau}} = \scal{\del^\dagger_{p+1} \ch{\sigma}}{\ch{\tau}}$.
    % \end{defn}

    % One can therefore think of the coboundary operator as a discrete exterior derivative acting on cochains.

\end{document}
