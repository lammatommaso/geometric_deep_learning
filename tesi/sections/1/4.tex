\documentclass[../1.tex]{subfiles}
\begin{document}

    In order to define data on simplicial complexes we are interested in studying the dual of the chains $\ket{\sigma}$,
    which we shall call cochains $\bra{\sigma}$.
       
    \begin{defn}
        Let $C_p$ be the group p-chains, we define the group of \ii{p-cochains} to 
        be $C^p := Hom(C_p, G)$, i.e. the dual space of $C_p$.
    \end{defn}

    As well as with for chains, we have also for cochains a sequence of homeomorphisms called \ii{coboundary maps}. The coboundary maps are defined
    to be the dual of the boundary maps, hence satisfying a dual version of the homology lemma called \ii{cohomology lemma}.

    \begin{defn}
        The dual of the boundary maps which we shall call
        \ii{coboundary maps}, is the group homomorphsm defined by
        \[ d_{p+1} : C^p \to C^{p+1} \quad d_p(\bra{\sigma}) := \bra{\sigma} \del_{p+1} \quad \forall \bra{\sigma} \in C^p.\]
        Therefore $(d_{p+1}\bra{\sigma})\ket{\tau} = \bra{\sigma} \del_{p+1} \ket{\tau} \quad \forall \ket{\tau} \in C_p$.
    \end{defn}

    The proof of $d^p+1 \circ d_p$ follow directly from the homology lemma. The cohomology lemma allows us to define the \ii{cohomology group}.
    
    \begin{defn}
        We define the \ii{p-cohomology group} to be 
        \[H^p := \frac{ker d_{p+1}}{im d_{p}},\] 
        where $im d_{p-1}$ is the group of simplicial p-cocycles and
        $ker d_p$ is the group of simplicial p-coboundaries.
    \end{defn}

    To have a specular intuition of what the coboundary operator actually does, we equipe our simplicial chain groups with
    an inner product. This finite dimensional Hilbert space structure shall allow us to represent the cochain $\bra{\sigma}$ evaluated on the chain $\ket{\tau}$ 
    as the inner product $\braket{\sigma}{\tau}$.

    \begin{defn}
        We define the scalar product $\braket{}{} : C_p \times C_p \to \R$ on the canonical basis of $C_p$ to be 
        \[ \braket{i}{j} = \delta_{ij} ,\]  
        where $i,j$ are any two p-simplexes in the canonical basis of $C_p$ and $\delta_{ij}$ is the Kronecker Delta.
    \end{defn}

    The same Hilbert space structure allows us to represent the coboundary operator $d$ acting on a cochain with its cochain representative.

    \begin{defn}
        Let $d_{p+1} : C^p \to C^{p+1}$ be the coboundary operator, then for any \\ $\bra{\sigma} \in C^p, \tau \in C_{p+1}$ we can define its dual representation $\del^\dagger_{p+1}$ by
        \[(d_{p+1}\bra{\sigma})\ket{\tau} = \bra{\sigma} \del_{p+1} \ket{\tau} = \bra{\tau} \del^\dagger_{p+1} \ket{\sigma} \quad \forall \ket{\tau} \in C_p. \]
    \end{defn}

    We can notice that our definitions lead to the restatement of the equivalent of the generalized Stokes' theorem on simplicial complexes
    according to the integration previously defined, i.e. $(d\bra{\sigma})\ket{\tau} = \bra{\sigma}\del\ket{\tau}$. One can therefore think of the coboundary
    operator as a discrete exterior derivative acting on cochains. Similarly we name the inner product \ii{integration} due to the analogy previously stated, the
    choice of a measure of integration is equivalent to a choice of a basis for the chain group.

\end{document}
