\documentclass[../3.tex]{subfiles}
\begin{document}
    While a linear combination of points of a domain is not necessarily defined, the space of signals on that domain can be equipped with a vector space
    structure. More formally we can define the space of $V$-valued signals on the domain $\Omega$.

    \begin{defn}
        Let $\Omega$ be a set
        , and $V$ a vector space over $\K$, we define the \ii{signal set}
        \[ \mc{S}(\Omega,V) = \{ \phi : \Omega \to V \}, \]
        where the dimensions of $V$ are called \ii{channels}.
    \end{defn}

    In particular cases such as when $\Omega$ is a smooth manifold we might also require the signals to be smooth.
    Notice that often $\Omega$ has an additional structure, for instance it is a manifold.
    Let us now equip the signal set with a vector space structure.

    The signal set equipped with the operations
    \[ \cdot : \K \times \mc{S}(\Omega,V) \to \mc{S}(\Omega,V)   \quad \text{such that} \; (\lambda \phi)(x) = \lambda \phi(x), \; \text{and} \]
    \[ + : \mc{S}(\Omega,V) \times \mc{S}(\Omega,V) \to \mc{S}(\Omega,V) \quad \text{such that} \; (\phi + \eta)(x) = \phi(x) + \eta(x), \]
    where $x \in \Omega$, is a vector space over $\K$. We leave to the reader the straight forward verification.


    We shall call this vector space \ii{signal space}, and denote it just by $\mc{S}$ for
    a lighter notation.

    Given an inner product $\langle \, , \, \rangle_V$ in $V$ and a measure $\mu$ on $\Omega$ the signal space can be equipped with a Hilbert space structure.

    \begin{defn}
        Let $\langle \, , \, \rangle_V : V \times V \to \K$ be an inner product and let $\mu$ be a measure on $\Omega$, then for any two signals
        $\phi, \eta \in \mc{S}$ we define their inner product $\langle \, , \, \rangle : \mc{S} \times \mc{S} \to \K$ as
        \[ \langle \phi, \eta \rangle := \int_\Omega d\mu(x) \langle \phi(x), \eta(x) \rangle_V .\]
    \end{defn}

    \begin{prop}
        The completion of the space $\mc{S}$ equipped with $\langle \, , \, \rangle$ is a Hilbert space.
    \end{prop}

    \begin{rem}
        When the domain $\Omega$ is discrete we can choose $\mu$ to be the counting measure, therefore turning the integral into a sum.
        Furthermore, if $\Omega$ is finite we do not need to take the completion in teh previous proposition.
    \end{rem}

\end{document}