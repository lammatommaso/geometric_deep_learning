\documentclass[../3.tex]{subfiles}
\begin{document}
    In Chapter \ref{ch:2} we defined the convolution of $f:\R^n \to \R$ with $g:\R^n \to \R$ as
    \[ (f * g)(x) = \int_{\R^n} dx' f(x')g(x-x'), \]
    in the literature of convolutional neural networks the function $f$ is called the \ii{input} of the
    convolutional layer, the functions $g$ is called \ii{filter} and the output $f*g$ is called \ii{feature map}.
    To give an interpretation of convolution we will discuss an example from \cite{deep}.
    Suppose we are tracking the location of a spaceship with a laser sensor. Our
    sensor gives a single output $x : \R \to \R^3$, where $x(t)$ is the position read by the sensor at the time t.
    Now suppose that our laser sensor is somewhat noisy. To obtain a less noisy
    measurement of the spaceship’s position, we average several measurements.
    Of course, more recent measurements are more relevant, so we will want this to
    be a weighted average that gives more weight to recent measurements. We can
    do this with a weighting function $w:\R \to \R$. If
    we apply such a weighted average operation at every moment, we obtain a new
    function s providing a smoothed estimate of the position of the spaceship:
    \[ s(t) = \int_\R da \, x(a)w(t-a), \]
    where $w(t-a)$ is the shifted weighting function.
    Since the aquisition rate of any instrument is not infinite, we need to see this convolution as discrete, by sampling it in a 1D grid,
    \[ s(t) = \sum_{a \in \Z} x(a)w(t-a), \]
    where $t$ is now an integer.
    Infinite measurements, even if countable, require an infinite amount of time to be completed, therefore a restriction of $\Z$
    into a subset $I$ is necessary. But what happens at the boundary of $I$?\\
    The translation of the weighting function has no meaning on the boundary of the time interval that we sampled, it is thus 
    a common practice to identify the two edges of the interval to restore the shift invariance of the domain.
    This might look as a pragmatic solution but it also tells us some truth about the symmetries of the task.
    If I need to classify a signal, e.g. voice recognition, the task is independent of the time when the signal was recorded,
    we can say that the classification is invariant under translations of the signal over time. In order to have a convolutional network that
    doesn't distinguish two time-shifted copies of the same signal we need a time-shift invariant \footnote{ Or equivariant.} convolution, and for this
    convolution to be well defined we need the domain of definition to be time-shift invariant too.
    Let's now go back to our example, where $I = {0,\ddots, n}$ we now write the convolution as
    \[ s_i = \sum_{j = 0}^n x_j w_{(i-j) \, \text{mod} \, n},\]
    where the "$\mod$" comes from the identification of the two edges of the interval.
    It looks like we are summing over the cyclic group $\Z_n$, which not by coincidence is also the symmetry group of our classification task.
    This gives rise to the question about the extensibility of convolution to general symmetry groups, as we shall see in the next section.
\end{document}