\documentclass[../3.tex]{subfiles}
\begin{document}
    The symmetries of an object are colloquially transformations that leave
    a certain \ii{classification} of said object unchanged or invariant. 

    \begin{defn}
        We define a \ii{classification} to be a function $C : \mc{S} \to \mc{L}$ from the signal space into some set $\mc{L}$.
    \end{defn}

    In many classifications task the set $\mc{L}$ is a label set.

    Using the concept of classification we could give a more formal definition of symmetry.

    \begin{defn}
        Let $C : \mc{S} \to \mc{L}$ be a classification we define a \ii{symmetry} of $C$ to be an invertible function
        $g : \mc{S} \to \mc{S}$ such that 
        \[ C[g(\phi)] = C[\phi] \quad \forall \phi \in \mc{S}. \]
    \end{defn}

    We will denite the set of these symmetries as $S_C := \{ g : \mc{S} \to \mc{S} \; : \; C[g(\phi)] = C[\phi] \;  \forall \phi \in \mc{S}\}$

    \begin{prop}
        Let $\circ$ be the composition of functions from $\mc{S}$ to itself, then $(S_C, \circ)$ is a group.
    \end{prop} 
    \begin{proof}
        We shall prove that this set with respect to $\circ$ satisfies all the group properties.
        \begin{enumerate}[(i)]
            \item The operation $\circ$ is already associative.
            \item Let $\text{id}_\mc{S} : \mc{S} \to \mc{S}$ be the identity on $\mc{S}$, namely $\text{id}_\mc{S}(\phi) = \phi$ for all $\phi \in \mc{S}$,
                then $\text{id}_\mc{S} \in S_C$.
            \item Let $g,g' \in S_C$, we have that $C[(g\circ g')(\phi)] = C[\phi]$, therefore $(g \circ g') \in S_C$.
            \item Let $g \in S_C$ then $C[g^{-1}(\phi)] = C[\phi]$, since $C[(g \circ g^{-1})(\phi)] = C[(\phi)] = C[g^{-1}(\phi)]$. \qedhere
        \end{enumerate}
    \end{proof}

    It is important to notice that in order to define a symmetry we need to choose a classification, namely there is no a priori symmetry in the data until
    we specify what we want to do with it.

    In order to deal with groups in an efficient way we need to define what the \ii{generators} of the group are.

    \begin{defn}
        Let $G$ be a group, and let $S \subset G$, if every element of $G$ can be written as composition of elements of $S$ and their invereses,
        then the elements of $S$ are called \ii{generators} of the group $G$.
    \end{defn}

\end{document}