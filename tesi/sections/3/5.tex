\documentclass[../3.tex]{subfiles}
\begin{document}

    Another approach that can be used to define a convolution is defining spectral convolution with respect to a laplacian operator, in this section
    we will see this abstract and general method on graphs.

    We start by recalling a classical result in Fourier analysis, the \ii{convolution theorem}.
    A proof of the theorem can be found in \cite{fourier} at chapter $7$.

    \begin{thm}
        Let $f,g \in L^1(\R,\C)$, and let $\mc{F}(f), \mc{F}(g) : \R \to \C$ be their respective Fourier transforms, then the fourier transform of the convolution of $f$ and $g$ is equal to
        \[ \mc{F}(f * g) = \mc{F}(f)\mc{F}(g) . \]
    \end{thm}  
    
    Let's now see how to use this analogy on graphs.

    In order to construct a convolutional neural network that classifies data defined on a graph we need a definition of convolution on graphs.
    First we introduce what we call \ii{graph Fourier transform}.
    
    \begin{defn}
        Let $\ket{f} \in C_0$ and $dimC_0 = n_0$, we define the \ii{graph Fourier transform}\\
        $\mc{F}_0 : \R^{n_0} \to \R^{n_0}$ to be
        \[ (\braket{i}{f})_{i \in I} \mapsto (\braket{e_i}{f})_{i \in I},\]
        where $I = \{1,...,n_0\}$ and $\ch{e_i}$ are the eigenfunctions of the $0$-laplacian.
    \end{defn}

    This transform only defines a change of basis since $\ket{f} = \sum_{i \in I}\braket{i}{f}\ket{i} = \sum_{i \in I}\braket{e_i}{f}\ket{e_i}$, and therefore is invertible.
    We can in fact represent the graph Fourier transform with the matrix $F^{-1}_{ij} = F^\dagger_{ij} := \braket{e_i}{j}$, and its inverse $F_{ij} := \braket{i}{e_j}$.
    To define a convolution between two $0$-chains we use the famous convolution theorem $\mc{F}(f * \psi) = \mc{F}(f)\mc{F}(\psi)$. 

    \begin{defn}
        Let $\{\ket{e_i}\}_{i \in I}$ be a basis of $0$-laplacian eigenchains, let $\ket{f},\ket{\psi} \in C_0$, we define the representatives of $\ket{f * \psi}$ on
        the laplacian eigenchains to be 
        \[ \braket{e_i}{f * \psi} := \braket{e_i}{f}\braket{e_i}{\psi} \quad \forall i \in I.\]
        Therefore $\ket{f * \psi} = \sum_{i \in I}\braket{e_i}{f}\braket{e_i}{\psi}\ket{e_i}$.
    \end{defn}

    This can be trivially extended to simplicial complexes and more generally on chain complexes using the eigenfunctions of higher dimensional laplacians, as done in \cite{simplicialNN}.

\end{document}
