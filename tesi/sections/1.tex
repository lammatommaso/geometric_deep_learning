\documentclass[../main.tex]{subfiles}

\begin{document}
     
    Given a point cloud in some $\R^n$ sampled from some meaningful subspace and the order of magnitude of the noise acting on the sample one can 
    construct a suitable set in $\R^n$, i.e. a \ii{geometric simplicial complex}, homotopy equivalent to the sampled set, therefore recovering some
    topological infromation about the set that was lost in the sampling. 
    Although this approach provides simplicial complexes with the topology inherited from the metric space it hides the power of simplicial complexes 
    to describe those networks and interactions which would happily live beside that topology, to make this distinction clear enough we will treat geometric 
    simplicial complexes as a realization of more abstract objects called \ii{abstract simplicial complexes}.

    \begin{section}{Simplicial complexes}
        {\color{red} In this section we shall define simplicial complexes, which, as abstract as they might look, can 
        be used to model interactions among individuals, traffic and road networks, as well as shapes, and 
        to approximate functions on some compact manifold.}
        \subfile{sections/1/1}    
    \end{section}
    \begin{section}{Forms and integration on abstract simplicial complexes}
        \subfile{sections/1/2}    
    \end{section}
    \begin{section}{The Laplace Operator}
    \end{section}
\end{document}
