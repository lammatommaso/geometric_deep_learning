\documentclass[12pt,a4paper]{report}
\usepackage[english]{babel}
\usepackage{newlfont}
\usepackage{color}
\textwidth=450pt\oddsidemargin=0pt
%\usepackage[utf8]{inputenc}
%\usepackage{fouriernc}
\usepackage[T1]{fontenc}
\usepackage{adjustbox}
\usepackage[margin=2cm]{geometry}
\usepackage{graphicx}
\usepackage{tikz}
\usepackage{hyperref}
\usepackage{amsmath}
\usepackage{amsthm}
\usepackage{amssymb}
\usepackage{mathtools}
\hypersetup{colorlinks=true,linkcolor=black,urlcolor=black, citecolor=black}
\usepackage{subfiles}
\usepackage{comment}
\usepackage{enumerate}
\usepackage[style=numeric, backend=biber, sorting=none]{biblatex}
\addbibresource{bibliography.bib}

\setlength{\parindent}{0pt}

\theoremstyle{plain}
\newtheorem{thm}{Theorem}[chapter]
\newtheorem{lem}[thm]{Lemma}
\newtheorem{prop}[thm]{Proposition}

\theoremstyle{definition}
\newtheorem{defn}[thm]{Definition}
\newtheorem{exa}[thm]{Example}

\theoremstyle{remark}
\newtheorem{rem}[thm]{Remark}

\setcounter{secnumdepth}{0} % no section numbering displayed, no idea how

\newcommand{\bb}[1]{\textbf{#1}}
\newcommand{\ii}[1]{\textit{#1}}
\newcommand{\eps}{\varepsilon}
\newcommand{\N}{\mathbb{N}}
\newcommand{\Z}{\mathbb{Z}}
\newcommand{\R}{\mathbb{R}}
\newcommand{\pder}[2]{\frac{\partial #1}{\partial #2}}
\newcommand{\eval}[1]{\Big{|}_{#1}}
\newcommand{\mc}[1]{\mathcal{#1}}
\newcommand{\mf}[1]{\mathfrak{#1}}
\newcommand{\bmc}[1]{\textbf\mathcal{#1}}
\newcommand{\scal}[2]{\langle #1, #2 \rangle}
\newcommand{\del}{\partial}

\begin{document}
    \begin{titlepage}
        \subfile{sections/front}
    \end{titlepage}
    \pagenumbering{roman}
    %   \subfile{sections/abstract_ita}
    %    \newpage
    %    \subfile{sections/abstract_eng}
    %    \newpage
    \begin{section}{Introduction}
        Most of the deep learning techniques used today are based on models 
        which learn a partition of the set of smooth functions defined on euclidean 
        domains into human friendly equivalence classes...
    \end{section}
    \tableofcontents
    \newpage
    \pagenumbering{arabic}
    \begin{chapter}{Preliminaries on topology}
        \subfile{sections/1}
    \end{chapter}
    \newpage
    \printbibliography[heading = bibintoc]
\end{document}
