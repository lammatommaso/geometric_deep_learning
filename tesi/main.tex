\documentclass[12pt,a4paper]{report}
\usepackage[english]{babel}
\usepackage{newlfont}
\usepackage{color}
\textwidth=450pt\oddsidemargin=0pt
%\usepackage[utf8]{inputenc}
%\usepackage{fouriernc}
\usepackage[T1]{fontenc}
\usepackage{adjustbox}
\usepackage[margin=2cm]{geometry}
\usepackage{graphicx}
\usepackage{wrapfig}
\usepackage{tikz}
\usepackage{hyperref}
\usepackage{amsmath}
\usepackage{amsthm}
\usepackage{amssymb}
\usepackage{mathtools}
\hypersetup{colorlinks=true,linkcolor=black,urlcolor=black, citecolor=black}
\usepackage{subfiles}
\usepackage{comment}
\usepackage{enumerate}
\usepackage[title]{appendix}
\usepackage[style=numeric, backend=biber, sorting=none]{biblatex}
\addbibresource{bibliography.bib}

\setlength{\parindent}{0pt}

\theoremstyle{plain}
\newtheorem{thm}{Theorem}[chapter]
\newtheorem{lem}[thm]{Lemma}
\newtheorem{prop}[thm]{Proposition}
\newtheorem{cor}[thm]{Corollary}

\theoremstyle{definition}
\newtheorem{defn}[thm]{Definition}
\newtheorem{exa}[thm]{Example}

\theoremstyle{remark}
\newtheorem{rem}[thm]{Remark}

\setcounter{secnumdepth}{0} % no section numbering displayed, no idea how

\newcommand{\bb}[1]{\textbf{#1}}
\newcommand{\ii}[1]{\textit{#1}}
\newcommand{\eps}{\varepsilon}
\newcommand{\N}{\mathbb{N}}
\newcommand{\Z}{\mathbb{Z}}
\newcommand{\R}{\mathbb{R}}
\newcommand{\pder}[2]{\frac{\partial #1}{\partial #2}}
\newcommand{\eval}[1]{\Big{|}_{#1}}
\newcommand{\mc}[1]{\mathcal{#1}}
\newcommand{\mf}[1]{\mathfrak{#1}}
\newcommand{\bmc}[1]{\textbf\mathcal{#1}}
\newcommand{\scal}[2]{\langle #1, #2 \rangle}
\newcommand{\bra}[1]{\langle #1 |}
\newcommand{\ket}[1]{| #1 \rangle}
\newcommand{\braket}[2]{\langle #1 | #2 \rangle}
\newcommand{\del}{\partial}

\begin{document}
    \begin{titlepage}
        \subfile{sections/front}
    \end{titlepage}
    \pagenumbering{roman}
    %   \subfile{sections/abstract_ita}
    %    \newpage
    %    \subfile{sections/abstract_eng}
    %    \newpage
    \begin{section}{Introduction}
        Most of the deep learning techniques used today are based on models which learn a partition of the set of smooth functions defined on euclidean 
        domains into human friendly equivalence classes. Although this approach has been successful in modern machine learning, it only deals with a really 
        small set of domains. The goal of geometric deep learning is to extend this method to data defined on manifolds and simplicial complexes.\\
        Convolution on euclidean domains is itself based on the translation invariance of such domains. In fact the convolution of a function $f : \R^n \to \R$,
        with some filter $g:\R^n \to \R$ is $(f * g)(x) = \scal{f}{g \circ T^{-1}}_{L^2}$, where $T:\R^n \to \R^n$ is a translation represented by the vector $x$.
        % Luckily convolutions can be define for functions define on groups with a measure, therefore we could also see the $\R^n$ with the measure $dx$ as a representation of the
        % group of translations $\bb{T}$ with measure $dT$, i.e. $(f * g)(x) = \int_{\bb{T}}dTf(T)g(x-T)$.
    \end{section}
    \tableofcontents
    \newpage
    \pagenumbering{arabic}
    \begin{chapter}{Preliminaries on topology}
        \subfile{sections/1}
    \end{chapter}
    \begin{chapter}{Simplicial Neural Networks}
        \subfile{sections/2}
    \end{chapter}
    % \begin{appendices}
    %     \begin{chapter}{Category Theory}
    %         \subfile{sections/appendixA/main}
    %     \end{chapter}
    % \end{appendices}
    \newpage
    \printbibliography[heading = bibintoc]
\end{document}
