\documentclass{article}
\usepackage[utf8]{inputenc}
\usepackage{fouriernc}
\usepackage[T1]{fontenc}
\usepackage{adjustbox}
\usepackage[margin=2cm]{geometry}
\usepackage{graphicx}
\usepackage{hyperref}
\usepackage{amsmath}
\usepackage{amsthm}
\usepackage{amssymb}
\usepackage{mathtools}
\hypersetup{colorlinks=true,linkcolor=blue,urlcolor=red}

\setlength{\parindent}{0pt}

\newtheorem{thm}{Theorem}[section]
\newtheorem*{thm*}{Theorem}
\newtheorem{prop}[thm]{Proposition}
\newtheorem*{prop*}{Proposition}
\newtheorem{defn}{Definition}[section]
\newtheorem*{defn*}{Definition}
\newtheorem{lem}[thm]{Lemma}
\newtheorem{cor}{Corollary}[thm]
\newtheorem{exa}{Example}
\newtheorem{exe}{Exercise}

\newcommand{\bb}[1]{\textbf{#1}}
\newcommand{\eps}{\varepsilon}
\newcommand{\N}{\mathbb{N}}
\newcommand{\R}{\mathbb{R}}
\newcommand{\pder}[2]{\frac{\partial #1}{\partial #2}}
\newcommand{\eval}[1]{\Big{|}_{#1}}

\title{Geometric Deep Learning \\ Beyond Euclidean Domains}
\author{}
\date{}
\begin{document}

\maketitle
\begin{section}{Geometric Priors}

In this section we want to define the mathematical foundation of deep learning via CNN'S.

\begin{defn}\underline{Our compact euclidean domain $\Omega$}\\
$\Omega := \prod_{i \in I}[0,1]$.
\end{defn}

\begin{defn}\underline{Classification}\\
Let $x \in L^2 := L^2(\Omega)$ then $f : L^2 \to \mathcal{C}$ surjective
is said to be a classification of $L^2$ on the set $\mathcal{C}$.
\end{defn}

\begin{defn}\underline{Training Set}\\
Let $f$ be a classification of $L^2$ on $\mathcal{C}$ and $\{x_i\}_{i \in I}\subset L^2$
then the set $\{(x_i,f(x_i))\}_{i \in I}$ is called a training set for $f$.
\end{defn}

\begin{prop}\underline{$card(\mathcal{C})<card(L^2(\Omega))$}\\
Let $f$ be a classification of $L^2$ on $\mathcal{C}$ then, given the inevitable noise acting on data,
there exists a real positive $\eps$ such that $\forall (x,x_\eps) \in L^2\times L^2: \int\limits_\Omega{|x-x_\eps|^2}<\eps$
we have that $f(x)=f(x_\eps)$.
\end{prop}

Given ideal data classification we can define two function $f$-equivalent if and only if 
their images via the classification $f$ are equal according to an equivalence on $\mathcal{C}$.

\begin{thm}\underline{$\simeq$ is an equivalence relation}\\
Let $x,y,z \in L^2$ we define $x\simeq y \iff f(x)=f(y)$ where $f$ is a classification of $L^2$ on $\mathcal{C}$,
then:\\
(i) $x \simeq x$\\
(ii) $x \simeq y \iff y \simeq x$\\
(iii) $x \simeq y, y \simeq z \implies x \simeq z$
\end{thm}
\begin{proof}
(i),(ii) and (iii) follow from the the equivalence on $\mathcal{C}$ by which they are defined.
\end{proof}



\end{section}

\end{document}
